% !TEX encoding = UTF-8
% !TEX TS-program = pdflatex
% !TEX root = ../tesi.tex

%**************************************************************
\chapter{Analisi dei requisiti}
\label{cap:analisi-requisiti}
%**************************************************************

\noindent \intro{Breve introduzione al capitolo}\\

\section{Casi d'uso}

\noindent Per lo studio dei casi di utilizzo del prodotto sono stati creati dei diagrammi.
I diagrammi dei casi d'uso (in inglese \emph{Use Case Diagram}) sono diagrammi di tipo \gls{umlg} dedicati alla descrizione delle funzioni o servizi offerti da un sistema, così come sono percepiti e utilizzati dagli attori che interagiscono col sistema stesso.
Essendo il progetto finalizzato alla creazione di un tool per l'automazione di un processo, le interazioni da parte dell'utilizzatore devono essere ovviamente ridotte allo stretto necessario. Per questo motivo i diagrammi dei casi d'uso risultano semplici e in numero ridotto.\\

\noindent A livello formale, i diagrammi dei casi d'uso avranno la seguente forma:
\begin{center}
    \textbf{UC<CodicePadre>.<CodiceFiglio>}
\end{center}
\noindent È importante ribadire come questo formalismo sia gerarchico, ovvero un codice figlio
può essere codice padre di un suo eventuale codice figlio. Possono essere figli le generalizzazioni e i sottocasi d'uso.\\

\noindent Nella figura illustrata di seguito verrà illustrato il diagramma del sistema principale con tutti i casi d'uso.
\begin{figure}[!h] 
    \centering 
    \includegraphics[width=0.9\columnwidth]{usecase/scenario-principale} 
    \caption{Use Case - UC0: Scenario principale}
\end{figure}

\noindent \textbf{\large UC 1 - Inserimento dati}
\label{uc:inserimento-dati}
\begin{itemize}

	\item \textbf{Attori primari: }
		\begin{itemize}
			\item Utente.
		\end{itemize}

	\item \textbf{Precondizione: }\\[0.3cm]
		L'utente è dentro la \textit{form} e non ha ancora inserito alcun dato.

	\item \textbf{Scenario principale: }
		\begin{enumerate}
			\item L'utente inserisce i dati.
		\end{enumerate}
		

	\item \textbf{Postcondizione: }\\[0.3cm]
		L'utente ha inserito i dati correttamente.

\end{itemize}

\vspace{0.5cm}


\noindent \textbf{\large UC 1.1 - Inserimento data di inizio previsione}
\label{uc:inserimento-data-inizio-prev}
\begin{itemize}

	\item \textbf{Attori primari: }
		\begin{itemize}
			\item Utente.
		\end{itemize}

	\item \textbf{Precondizione: }\\[0.3cm]
		L'utente è dentro la \textit{form} e non ha ancora inserito la data di inizio previsione.

	\item \textbf{Scenario principale: }
		\begin{enumerate}
			\item L'utente seleziona la data di inizio previsione.
		\end{enumerate}

	\item \textbf{Postcondizione: }\\[0.3cm]
		L'utente ha inserito la data di inizio previsione correttamente.

	\item \textbf{Scenario alternativo: }
		\begin{itemize}
		    \item La \textit{form} segnala un errore di immissione dati.
		\end{itemize}

\end{itemize}

\vspace{0.5cm}

\noindent \textbf{\large UC 1.2 - Inserimento data di fine previsione}
\label{uc:inserimento-data-fine-prev}
\begin{itemize}

	\item \textbf{Attori primari: }
		\begin{itemize}
			\item Utente.
		\end{itemize}

	\item \textbf{Precondizione: }\\[0.3cm]
		L'utente è dentro la \textit{form} e non ha ancora inserito la data di fine previsione.

	\item \textbf{Scenario principale: }
		\begin{enumerate}
			\item L'utente inserisce la data di fine previsione;
		\end{enumerate}

	\item \textbf{Postcondizione: }\\[0.3cm]
		L'utente ha inserito la data di fine previsione correttamente.

	\item \textbf{Scenario alternativo: }
		\begin{itemize}
		    \item La \textit{form} segnala un errore di immissione dati.
		\end{itemize}

\end{itemize}

\vspace{0.5cm}

\noindent \textbf{\large UC 1.3 - Errore: data di inizio previsione antecedente alla data odierna}
\label{uc:err-inserimento-data-inizio-prev}
\begin{itemize}

	\item \textbf{Attori primari: }
		\begin{itemize}
			\item Utente.
		\end{itemize}

	\item \textbf{Precondizione: }\\[0.3cm]
		L'utente è dentro la \textit{form} e ha inserito una data di inizio previsione antecedente
		alla data odierna.

	\item \textbf{Scenario principale: }
		\begin{enumerate}
			\item L'utente inserisce la data di inizio previsione;
			\item L'utente visualizza un errore generato dalla \textit{form}.
		\end{enumerate}
		

	\item \textbf{Postcondizione: }\\[0.3cm]
		L'utente viene avvisato dell'errore di immissione.

\end{itemize}

\vspace{0.5cm}

\noindent \textbf{\large UC 1.4 - Errore: data di fine previsione antecedente alla data di \\\hspace*{56pt}inizio previsione}
\label{uc:err-inserimento-fine-fine-prev}
\begin{itemize}

	\item \textbf{Attori primari: }
		\begin{itemize}
			\item Utente.
		\end{itemize}

	\item \textbf{Precondizione: }\\[0.3cm]
		L'utente è dentro la \textit{form} e ha inserito una data di fine previsione antecedente
		alla data odierna.

	\item \textbf{Scenario principale: }
		\begin{enumerate}
			\item L'utente inserisce la data di fine previsione;
			\item L'utente visualizza un errore generato dalla \textit{form}.
		\end{enumerate}
		

	\item \textbf{Postcondizione: }\\[0.3cm]
		L'utente viene avvisato dell'errore di immissione.

\end{itemize}

\vspace{0.5cm}

\noindent \textbf{\large UC 2 - Inizio analisi di ottimizzazione}
\label{uc:inizio-analisi-ottimizzazione}
\begin{itemize}

	\item \textbf{Attori primari: }
		\begin{itemize}
			\item Utente.
		\end{itemize}

	\item \textbf{Precondizione: }\\[0.3cm]
		L'utente è dentro la \textit{form} e ha inserito una data di inizio e fine previsione valide.

	\item \textbf{Scenario principale: }
		\begin{enumerate}
			\item L'utente conferma l'inizio dell'analisi di ottimizzazione.
		\end{enumerate}
		

	\item \textbf{Postcondizione: }\\[0.3cm]
		L'utente ha effettuato l'analisi di ottimizzazione per le date di inizio e fine previsione e visualizza correttamente i risultati.

\end{itemize}

\vspace{0.5cm}

\noindent \textbf{\large UC 3 - Visualizzazione risultati}
\label{uc:visualizzazione-risultati}
\begin{itemize}

	\item \textbf{Attori primari: }
		\begin{itemize}
			\item Utente.
		\end{itemize}

	\item \textbf{Precondizione: }\\[0.3cm]
		L'utente è dentro la \textit{form} e ha effettuato l'analisi di ottimizzazione correttamente.

	\item \textbf{Scenario principale: }
		\begin{enumerate}
			\item L'utente visualizza il totale non ottimizzato;
			\item L'utente visualizza il totale ottimizzato;
			\item L'utente visualizza lo scostamento percentuale dei totali.
		\end{enumerate}
		

	\item \textbf{Postcondizione: }\\[0.3cm]
		L'utente visualizza correttamente tutti i risultati.

\end{itemize}

\vspace{0.5cm}

\noindent \textbf{\large UC 3.1 - Visualizzazione totale non ottimizzato}
\label{uc:visualizzazione-totale-non-ottimizzato}
\begin{itemize}

	\item \textbf{Attori primari: }
		\begin{itemize}
			\item Utente.
		\end{itemize}

	\item \textbf{Precondizione: }\\[0.3cm]
		L'utente è dentro la \textit{form} e ha effettuato l'analisi di ottimizzazione correttamente.

	\item \textbf{Scenario principale: }
		\begin{enumerate}
			\item L'utente visualizza il totale non ottimizzato.
		\end{enumerate}
		

	\item \textbf{Postcondizione: }\\[0.3cm]
		L'utente visualizza correttamente il totale non ottimizzato.

\end{itemize}

\vspace{0.5cm}

\noindent \textbf{\large UC 3.2 - Visualizzazione totale ottimizzato}
\label{uc:visualizzazione-totale-ottimizzato}
\begin{itemize}

	\item \textbf{Attori primari: }
		\begin{itemize}
			\item Utente.
		\end{itemize}

	\item \textbf{Precondizione: }\\[0.3cm]
		L'utente è dentro la \textit{form} e ha effettuato l'analisi di ottimizzazione correttamente.

	\item \textbf{Scenario principale: }
		\begin{enumerate}
			\item L'utente visualizza il totale ottimizzato.
		\end{enumerate}
		

	\item \textbf{Postcondizione: }\\[0.3cm]
		L'utente visualizza correttamente il totale ottimizzato.

\end{itemize}

\vspace{0.5cm}

\noindent \textbf{\large UC 3.3 - Visualizzazione scostamento percentuale dei totali}
\label{uc:visualizzazione-scostamento-percentuale-totali}
\begin{itemize}

	\item \textbf{Attori primari: }
		\begin{itemize}
			\item Utente.
		\end{itemize}

	\item \textbf{Precondizione: }\\[0.3cm]
		L'utente è dentro la \textit{form} e ha effettuato l'analisi di ottimizzazione correttamente.

	\item \textbf{Scenario principale: }
		\begin{enumerate}
			\item L'utente visualizza lo scostamento percentuale dei totali.
		\end{enumerate}
		

	\item \textbf{Postcondizione: }\\[0.3cm]
		L'utente visualizza correttamente lo scostamento percentuale dei totali.

\end{itemize}

\vspace{0.5cm}

\noindent \textbf{\large UC 4 - Visualizzazione lista degli ordini}
\label{uc:visualizzazione-lista-ordini}
\begin{itemize}

	\item \textbf{Attori primari: }
		\begin{itemize}
			\item Utente.
		\end{itemize}

	\item \textbf{Precondizione: }\\[0.3cm]
		L'utente è dentro la \textit{form} e ha effettuato l'analisi di ottimizzazione correttamente.

	\item \textbf{Scenario principale: }
		\begin{enumerate}
			\item L'utente visualizza la lista degli ordini da effettuare.
		\end{enumerate}
		

	\item \textbf{Postcondizione: }\\[0.3cm]
		L'utente visualizza correttamente la lista degli ordini da effettuare.

\end{itemize}

\vspace{0.5cm}

\noindent \textbf{\large UC 4.1 - Visualizzazione singolo ordine}
\label{uc:visualizzazione-singolo-ordine}
\begin{itemize}

	\item \textbf{Attori primari: }
		\begin{itemize}
			\item Utente.
		\end{itemize}

	\item \textbf{Precondizione: }\\[0.3cm]
		L'utente visualizza correttamente la lista degli ordini.

	\item \textbf{Scenario principale: }
		\begin{enumerate}
			\item L'utente visualizza il singolo ordine con tutte le informazioni.
		\end{enumerate}
		

	\item \textbf{Postcondizione: }\\[0.3cm]
		L'utente visualizza correttamente il singolo ordine.

\end{itemize}

\vspace{0.5cm}

\noindent \textbf{\large UC 4.1.1 - Visualizzazione codice articolo dell'ordine}
\label{uc:visualizzazione-codice-articolo}
\begin{itemize}

	\item \textbf{Attori primari: }
		\begin{itemize}
			\item Utente.
		\end{itemize}

	\item \textbf{Precondizione: }\\[0.3cm]
		L'utente visualizza correttamente il singolo ordine.

	\item \textbf{Scenario principale: }
		\begin{enumerate}
			\item L'utente visualizza il codice articolo del singolo ordine.
		\end{enumerate}
		

	\item \textbf{Postcondizione: }\\[0.3cm]
		L'utente visualizza correttamente il codice articolo del singolo ordine.

\end{itemize}

\vspace{0.5cm}

\noindent \textbf{\large UC 4.1.2 - Visualizzazione codice fornitore dell'ordine}
\label{uc:visualizzazione-codice-fornitore}
\begin{itemize}

	\item \textbf{Attori primari: }
		\begin{itemize}
			\item Utente.
		\end{itemize}

	\item \textbf{Precondizione: }\\[0.3cm]
		L'utente visualizza correttamente il singolo ordine.

	\item \textbf{Scenario principale: }
		\begin{enumerate}
			\item L'utente visualizza il codice fornitore del singolo ordine.
		\end{enumerate}
		

	\item \textbf{Postcondizione: }\\[0.3cm]
		L'utente visualizza correttamente il codice fornitore del singolo ordine.

\end{itemize}

\vspace{0.5cm}

\noindent \textbf{\large UC 4.1.3 - Visualizzazione data d'ordine dell'ordine}
\label{uc:visualizzazione-data-ordine}
\begin{itemize}

	\item \textbf{Attori primari: }
		\begin{itemize}
			\item Utente.
		\end{itemize}

	\item \textbf{Precondizione: }\\[0.3cm]
		L'utente visualizza correttamente il singolo ordine.

	\item \textbf{Scenario principale: }
		\begin{enumerate}
			\item L'utente visualizza la data d'ordine del singolo ordine.
		\end{enumerate}
		

	\item \textbf{Postcondizione: }\\[0.3cm]
		L'utente visualizza correttamente la data d'ordine del singolo ordine.

\end{itemize}

\vspace{0.5cm}

\noindent \textbf{\large UC 4.1.4 - Visualizzazione data iniziale di copertura dell'ordine}
\label{uc:visualizzazione-data-iniziale-copertura}
\begin{itemize}

	\item \textbf{Attori primari: }
		\begin{itemize}
			\item Utente.
		\end{itemize}

	\item \textbf{Precondizione: }\\[0.3cm]
		L'utente visualizza correttamente il singolo ordine.

	\item \textbf{Scenario principale: }
		\begin{enumerate}
			\item L'utente visualizza la data iniziale di copertura del singolo ordine.
		\end{enumerate}
		

	\item \textbf{Postcondizione: }\\[0.3cm]
		L'utente visualizza correttamente la data iniziale di copertura del singolo ordine.

\end{itemize}

\vspace{0.5cm}

\noindent \textbf{\large UC 4.1.5 - Visualizzazione data finale di copertura dell'ordine}
\label{uc:visualizzazione-data-finale-copertura}
\begin{itemize}

	\item \textbf{Attori primari: }
		\begin{itemize}
			\item Utente.
		\end{itemize}

	\item \textbf{Precondizione: }\\[0.3cm]
		L'utente visualizza correttamente il singolo ordine.

	\item \textbf{Scenario principale: }
		\begin{enumerate}
			\item L'utente visualizza la data finale di copertura del singolo ordine.
		\end{enumerate}
		

	\item \textbf{Postcondizione: }\\[0.3cm]
		L'utente visualizza correttamente la data finale di copertura del singolo ordine.

\end{itemize}

\vspace{0.5cm}

\noindent \textbf{\large UC 4.1.6 - Visualizzazione quantità ordinata dell'ordine}
\label{uc:visualizzazione-quantità-ordinata}
\begin{itemize}

	\item \textbf{Attori primari: }
		\begin{itemize}
			\item Utente.
		\end{itemize}

	\item \textbf{Precondizione: }\\[0.3cm]
		L'utente visualizza correttamente il singolo ordine.

	\item \textbf{Scenario principale: }
		\begin{enumerate}
			\item L'utente visualizza la quantità ordinata del singolo ordine.
		\end{enumerate}
		

	\item \textbf{Postcondizione: }\\[0.3cm]
		L'utente visualizza correttamente la quantità ordinata del singolo ordine.

\end{itemize}

\vspace{0.5cm}

\noindent \textbf{\large UC 4.1.7 - Visualizzazione prezzo totale dell'ordine}
\label{uc:visualizzazione-prezzo-totale-ord}
\begin{itemize}

	\item \textbf{Attori primari: }
		\begin{itemize}
			\item Utente.
		\end{itemize}

	\item \textbf{Precondizione: }\\[0.3cm]
		L'utente visualizza correttamente il singolo ordine.

	\item \textbf{Scenario principale: }
		\begin{enumerate}
			\item L'utente visualizza il prezzo totale del singolo ordine.
		\end{enumerate}
		

	\item \textbf{Postcondizione: }\\[0.3cm]
		L'utente visualizza correttamente il prezzo totale del singolo ordine.

\end{itemize}

\vspace{0.5cm}

\noindent \textbf{\large UC 5 - Filtraggio dati}
\label{uc:filtraggio-dati}
\begin{itemize}

	\item \textbf{Attori primari: }
		\begin{itemize}
			\item Utente.
		\end{itemize}

	\item \textbf{Precondizione: }\\[0.3cm]
		L'utente visualizza correttamente la lista degli ordini.

	\item \textbf{Scenario principale: }
		\begin{enumerate}
			\item L'utente sceglie un filtro da applicare alla lista.
		\end{enumerate}
		

	\item \textbf{Postcondizione: }\\[0.3cm]
		L'utente visualizza correttamente tutti gli elementi che soddisfano il filtro.

\end{itemize}

\vspace{0.5cm}

\noindent \textbf{\large UC 5.1 - Filtraggio per ricerca generica}
\label{uc:filtraggio-ricerca-generica}
\begin{itemize}

	\item \textbf{Attori primari: }
		\begin{itemize}
			\item Utente.
		\end{itemize}

	\item \textbf{Precondizione: }\\[0.3cm]
		L'utente visualizza correttamente la lista degli ordini.

	\item \textbf{Scenario principale: }
		\begin{enumerate}
			\item L'utente filtra uno o più ordini tramite una ricerca.
		\end{enumerate}
		

	\item \textbf{Postcondizione: }\\[0.3cm]
		L'utente visualizza correttamente tutti gli elementi che soddisfano il filtro.

\end{itemize}

\vspace{0.5cm}

\noindent \textbf{\large UC 5.2 - Filtraggio per codice articolo}
\label{uc:filtraggio-codice-articolo}
\begin{itemize}

	\item \textbf{Attori primari: }
		\begin{itemize}
			\item Utente.
		\end{itemize}

	\item \textbf{Precondizione: }\\[0.3cm]
		L'utente visualizza correttamente la lista degli ordini.

	\item \textbf{Scenario principale: }
		\begin{enumerate}
			\item L'utente filtra uno o più ordini per codice articolo.
		\end{enumerate}
		

	\item \textbf{Postcondizione: }\\[0.3cm]
		L'utente visualizza correttamente tutti gli elementi che soddisfano il filtro.

\end{itemize}

\vspace{0.5cm}

\noindent \textbf{\large UC 5.3 - Filtraggio per codice fornitore}
\label{uc:filtraggio-codice-fornitore}
\begin{itemize}

	\item \textbf{Attori primari: }
		\begin{itemize}
			\item Utente.
		\end{itemize}

	\item \textbf{Precondizione: }\\[0.3cm]
		L'utente visualizza correttamente la lista degli ordini.

	\item \textbf{Scenario principale: }
		\begin{enumerate}
			\item L'utente filtra uno o più ordini per codice fornitore.
		\end{enumerate}
		

	\item \textbf{Postcondizione: }\\[0.3cm]
		L'utente visualizza correttamente tutti gli elementi che soddisfano il filtro.

\end{itemize}

\vspace{0.5cm}

\noindent \textbf{\large UC 5.4 - Filtraggio per data d'ordine}
\label{uc:filtraggio-data-ordine}
\begin{itemize}

	\item \textbf{Attori primari: }
		\begin{itemize}
			\item Utente.
		\end{itemize}

	\item \textbf{Precondizione: }\\[0.3cm]
		L'utente visualizza correttamente la lista degli ordini.

	\item \textbf{Scenario principale: }
		\begin{enumerate}
			\item L'utente filtra uno o più ordini per data d'ordine.
		\end{enumerate}
		

	\item \textbf{Postcondizione: }\\[0.3cm]
		L'utente visualizza correttamente tutti gli elementi che soddisfano il filtro.

\end{itemize}

\vspace{0.5cm}

\noindent \textbf{\large UC 5.5 - Filtraggio per data iniziale di copertura}
\label{uc:filtraggio-data-iniziale-copertura}
\begin{itemize}

	\item \textbf{Attori primari: }
		\begin{itemize}
			\item Utente.
		\end{itemize}

	\item \textbf{Precondizione: }\\[0.3cm]
		L'utente visualizza correttamente la lista degli ordini.

	\item \textbf{Scenario principale: }
		\begin{enumerate}
			\item L'utente filtra uno o più ordini per data iniziale di copertura.
		\end{enumerate}
		

	\item \textbf{Postcondizione: }\\[0.3cm]
		L'utente visualizza correttamente tutti gli elementi che soddisfano il filtro.

\end{itemize}

\vspace{0.5cm}

\noindent \textbf{\large UC 5.6 - Filtraggio per data finale di copertura}
\label{uc:filtraggio-data-finale-copertura}
\begin{itemize}

	\item \textbf{Attori primari: }
		\begin{itemize}
			\item Utente.
		\end{itemize}

	\item \textbf{Precondizione: }\\[0.3cm]
		L'utente visualizza correttamente la lista degli ordini.

	\item \textbf{Scenario principale: }
		\begin{enumerate}
			\item L'utente filtra uno o più ordini per data finale di copertura.
		\end{enumerate}
		

	\item \textbf{Postcondizione: }\\[0.3cm]
		L'utente visualizza correttamente tutti gli elementi che soddisfano il filtro.

\end{itemize}

\vspace{0.5cm}

\noindent \textbf{\large UC 5.7 - Filtraggio per quantità ordinata}
\label{uc:filtraggio-quantità-ordinata}
\begin{itemize}

	\item \textbf{Attori primari: }
		\begin{itemize}
			\item Utente.
		\end{itemize}

	\item \textbf{Precondizione: }\\[0.3cm]
		L'utente visualizza correttamente la lista degli ordini.

	\item \textbf{Scenario principale: }
		\begin{enumerate}
			\item L'utente filtra uno o più ordini per quantità ordinata.
		\end{enumerate}
		

	\item \textbf{Postcondizione: }\\[0.3cm]
		L'utente visualizza correttamente tutti gli elementi che soddisfano il filtro.

\end{itemize}

\vspace{0.5cm}

\noindent \textbf{\large UC 5.8 - Filtraggio per prezzo totale del singolo ordine}
\label{uc:filtraggio-prezzo-totale-ord}
\begin{itemize}

	\item \textbf{Attori primari: }
		\begin{itemize}
			\item Utente.
		\end{itemize}

	\item \textbf{Precondizione: }\\[0.3cm]
		L'utente visualizza correttamente la lista degli ordini.

	\item \textbf{Scenario principale: }
		\begin{enumerate}
			\item L'utente filtra uno o più ordini per prezzo totale del singolo ordine.
		\end{enumerate}
		

	\item \textbf{Postcondizione: }\\[0.3cm]
		L'utente visualizza correttamente tutti gli elementi che soddisfano il filtro.

\end{itemize}

\vspace{0.5cm}

\noindent \textbf{\large UC 6 - Ordinamento elementi della lista}
\label{uc:ordinamento-elementi-lista}
\begin{itemize}

	\item \textbf{Attori primari: }
		\begin{itemize}
			\item Utente.
		\end{itemize}

	\item \textbf{Precondizione: }\\[0.3cm]
		L'utente visualizza correttamente la lista degli ordini.

	\item \textbf{Scenario principale: }
		\begin{enumerate}
			\item L'utente sceglie l'ordinamento da applicare alla lista.
		\end{enumerate}
		

	\item \textbf{Postcondizione: }\\[0.3cm]
		L'utente visualizza correttamente tutti gli elementi ordinati secondo la sua scelta.
    
    \item \textbf{Generalizzazioni: }
        \begin{itemize}
            \item Ordinamento per codice articolo;
            \item Ordinamento per codice fornitore;
            \item Ordinamento per data d'ordine;
            \item Ordinamento per data previsione inizio copertura;
            \item data previsione fine copertura;
            \item quantità ordinata;
            \item prezzo totale del singolo ordine.
        \end{itemize}
\end{itemize}

\vspace{0.5cm}

\noindent \textbf{\large UC 6.1 - Ordinamento per codice articolo }
\label{uc:ordinamento-codice-articolo}
\begin{itemize}

	\item \textbf{Attori primari: }
		\begin{itemize}
			\item Utente.
		\end{itemize}

	\item \textbf{Precondizione: }\\[0.3cm]
		L'utente visualizza correttamente la lista degli ordini.
\item \textbf{Scenario principale: }
		\begin{enumerate}
			\item L'utente ordina la lista per codice articolo.
		\end{enumerate}
		

	\item \textbf{Postcondizione: }\\[0.3cm]
		L'utente visualizza correttamente tutti gli elementi che soddisfano il filtro.

\end{itemize}

\vspace{0.5cm}

\noindent \textbf{\large UC 6.2 - Ordinamento per codice fornitore }
\label{uc:ordinamento-codice-fornitore}
\begin{itemize}

	\item \textbf{Attori primari: }
		\begin{itemize}
			\item Utente.
		\end{itemize}

	\item \textbf{Precondizione: }\\[0.3cm]
		L'utente visualizza correttamente la lista degli ordini.

	\item \textbf{Scenario principale: }
		\begin{enumerate}
			\item L'utente ordina la lista per codice fornitore.
		\end{enumerate}
		

	\item \textbf{Postcondizione: }\\[0.3cm]
		L'utente visualizza correttamente tutti gli elementi che soddisfano il filtro.

\end{itemize}

\vspace{0.5cm}

\noindent \textbf{\large UC 6.3 - Ordinamento per data d'ordine }
\label{uc:ordinamento-data-ordine}
\begin{itemize}

	\item \textbf{Attori primari: }
		\begin{itemize}
			\item Utente.
		\end{itemize}

	\item \textbf{Precondizione: }\\[0.3cm]
		L'utente visualizza correttamente la lista degli ordini.

	\item \textbf{Scenario principale: }
		\begin{enumerate}
			\item L'utente ordina la lista per data d'ordine.
		\end{enumerate}
		

	\item \textbf{Postcondizione: }\\[0.3cm]
		L'utente visualizza correttamente tutti gli elementi che soddisfano il filtro.

\end{itemize}

\vspace{0.5cm}

\noindent \textbf{\large UC 6.4 - Ordinamento per data iniziale di copertura }
\label{uc:ordinamento-data-iniziale-copertura}
\begin{itemize}

	\item \textbf{Attori primari: }
		\begin{itemize}
			\item Utente.
		\end{itemize}

	\item \textbf{Precondizione: }\\[0.3cm]
		L'utente visualizza correttamente la lista degli ordini.

	\item \textbf{Scenario principale: }
		\begin{enumerate}
			\item L'utente filtra uno o più ordini per data iniziale di copertura.
		\end{enumerate}
		

	\item \textbf{Postcondizione: }\\[0.3cm]
		L'utente visualizza correttamente tutti gli elementi che soddisfano il filtro.

\end{itemize}

\vspace{0.5cm}

\noindent \textbf{\large UC 6.5 - Ordinamento per data finale di copertura}
\label{uc:ordinamento-data-finale-copertura}
\begin{itemize}

	\item \textbf{Attori primari: }
		\begin{itemize}
			\item Utente.
		\end{itemize}

	\item \textbf{Precondizione: }\\[0.3cm]
		L'utente visualizza correttamente la lista degli ordini.

	\item \textbf{Scenario principale: }
		\begin{enumerate}
			\item L'utente filtra uno o più ordini per data finale di copertura.
		\end{enumerate}
		

	\item \textbf{Postcondizione: }\\[0.3cm]
		L'utente visualizza correttamente tutti gli elementi che soddisfano il filtro.

\end{itemize}

\vspace{0.5cm}

\noindent \textbf{\large UC 6.6 - Ordinamento per quantità ordinata}
\label{uc:ordinamento-quantità-ordinata}
\begin{itemize}

	\item \textbf{Attori primari: }
		\begin{itemize}
			\item Utente.
		\end{itemize}

	\item \textbf{Precondizione: }\\[0.3cm]
		L'utente visualizza correttamente la lista degli ordini.

	\item \textbf{Scenario principale: }
		\begin{enumerate}
			\item L'utente filtra uno o più ordini per quantità ordinata.
		\end{enumerate}
		

	\item \textbf{Postcondizione: }\\[0.3cm]
		L'utente visualizza correttamente tutti gli elementi che soddisfano il filtro.

\end{itemize}

\vspace{0.5cm}

\noindent \textbf{\large UC 6.7 - Ordinamento per prezzo totale del singolo ordine}
\label{uc:ordinamento-prezzo-totale-ord}
\begin{itemize}

	\item \textbf{Attori primari: }
		\begin{itemize}
			\item Utente.
		\end{itemize}

	\item \textbf{Precondizione: }\\[0.3cm]
		L'utente visualizza correttamente la lista degli ordini.

	\item \textbf{Scenario principale: }
		\begin{enumerate}
			\item L'utente ordina gli elementi.
		\end{enumerate}
		

	\item \textbf{Postcondizione: }\\[0.3cm]
		L'utente visualizza correttamente tutti gli elementi rispetto al prezzo totale del singolo ordine.

\end{itemize}

\vspace{0.5cm}

\section{Tracciamento dei requisiti}

\noindent Da un'attenta analisi dei requisiti e degli use case effettuata sul progetto è stata stilata la tabella che traccia i requisiti in rapporto agli use case.\\
Sono stati individuati diversi tipi di requisiti e si è dunque utilizzato un codice identificativo univoco per distinguerli.\\
Il codice dei requisiti è così strutturato:
\begin{center}
    \textbf{R<NumeroRequisito>-<Tipo>-<Classificazione>}
\end{center}
In particolare il tipo può assumere 4 valori, quali:
\begin{itemize}
    \item \textbf{F} = funzionale
    \item \textbf{Q} = qualitativo
    \item \textbf{P} = performance
    \item \textbf{V} = vincolo
\end{itemize}
Per quanto riguarda la classificazione, invece, si hanno 3 valori possibili:
\begin{itemize}
    \item \textbf{O} = obbligatorio
    \item \textbf{D} = desiderabile
    \item \textbf{F} = facoltativo
\end{itemize}
Nelle tabelle \ref{tab:requisiti-funzionali}, \ref{tab:requisiti-qualitativi}, \ref{tab:requisiti-di-performance} \ref{tab:requisiti-di-vincolo} suddivise per tipo sono riassunti i requisiti e il loro tracciamento con gli use case delineati in fase di analisi.

%comando arraystretch
\renewcommand{\arraystretch}{1.6}

% tab funzionali

\begin{center}
%\rowcolors{2}{lightest-grayest}{white}
\begin{longtable}{|p{2cm}|p{9cm}|p{2cm}|}
\caption{Tabella del tracciamento dei requisiti funzionali}
\label{tab:requisiti-funzionali}
\\ \hline
\centering \textbf{Requisito} & \centering \textbf{Descrizione} & \centering \textbf{Use Case} \arraybackslash \\
\hline 
\req{R1-F-O}{L'utente deve poter inserire i dati necessari per l'ottimizzazione}{\hyperref[uc:inserimento-dati]{UC1}}
\req{R2-F-O}{L'utente deve poter inserire la data di inizio previsione}{\hyperref[uc:inserimento-data-inizio-prev]{UC1.1}}
\req{R3-F-O}{L'utente deve poter inserire la data di fine previsione}{\hyperref[uc:inserimento-data-fine-prev]{UC1.2}}
\req{R4-F-O}{L'utente deve poter essere avvisato dell'errore di inserimento della data di inizio previsione}{\hyperref[uc:err-inserimento-data-inizio-prev]{UC1.3}}
\req{R5-F-O}{L'utente deve poter essere avvisato dell'errore di inserimento della data di fine previsione}{\hyperref[uc:err-inserimento-fine-fine-prev]{UC1.4}}
\req{R6-F-O}{L'utente deve poter iniziare l'analisi di ottimizzazione}{\hyperref[uc:inizio-analisi-ottimizzazione]{UC2}}
\req{R7-F-O}{L'utente deve poter visualizzare i risultati}{\hyperref[uc:visualizzazione-risultati]{UC3}}
\req{R8-F-O}{L'utente deve poter visualizzare il totale non ottimizzato}{\hyperref[uc:visualizzazione-totale-non-ottimizzato]{UC3.1}}
\req{R9-F-O}{L'utente deve poter visualizzare il totale ottimizzato}{\hyperref[uc:visualizzazione-totale-ottimizzato]{UC3.2}}
\req{R10-F-O}{L'utente deve poter visualizzare lo scostamento tra i totali}{\hyperref[uc:visualizzazione-scostamento-percentuale-totali]{UC3.3}}
\req{R11-F-O}{L'utente deve poter visualizzare la lista degli ordini in maniera decrescente rispetto al codice articolo}{\hyperref[uc:visualizzazione-lista-ordini]{UC4}}
\req{R12-F-O}{L'utente deve poter visualizzare un singolo ordine della lista}{\hyperref[uc:visualizzazione-singolo-ordine]{UC4.1}}
\req{R13-F-O}{L'utente deve poter visualizzare il codice articolo di un ordine}{\hyperref[uc:visualizzazione-codice-articolo]{UC4.1.1}}
\req{R14-F-O}{L'utente deve poter visualizzare il codice fornitore di un ordine}{\hyperref[uc:visualizzazione-codice-fornitore]{UC4.1.2}}
\req{R15-F-O}{L'utente deve poter visualizzare la data d'ordine di un ordine}{\hyperref[uc:visualizzazione-data-ordine]{UC4.1.3}}
\req{R16-F-O}{L'utente deve poter visualizzare la data iniziale di copertua di un ordine}{\hyperref[uc:visualizzazione-data-iniziale-copertura]{UC4.1.4}}
\req{R17-F-O}{L'utente deve poter visualizzare la data finale di copertura di un ordine}{\hyperref[uc:visualizzazione-data-finale-copertura]{UC4.1.5}}
\req{R18-F-O}{L'utente deve poter visualizzare la quantità ordinata di un ordine}{\hyperref[uc:visualizzazione-quantità-ordinata]{UC4.1.6}}
\req{R19-F-O}{L'utente deve poter visualizzare il prezzo totale di un ordine}{\hyperref[uc:visualizzazione-prezzo-totale-ord]{UC4.1.7}}
\req{R20-F-O}{L'utente deve poter filtrare la lista tramite una ricerca}{\hyperref[uc:filtraggio-dati]{UC5}}
\req{R21-F-O}{L'utente deve poter filtrare la lista tramite una ricerca generica}{\hyperref[uc:filtraggio-ricerca-generica]{UC5.1}}
\req{R22-F-O}{L'utente deve poter filtrare la lista per codice articolo}{\hyperref[uc:filtraggio-codice-articolo]{UC5.2}}
\req{R23-F-O}{L'utente deve poter filtrare la lista per codice fornitore}{\hyperref[uc:filtraggio-codice-fornitore]{UC5.3}}
\req{R24-F-O}{L'utente deve poter filtrare la lista per data d'ordine}{\hyperref[uc:filtraggio-data-ordine]{UC5.4}}
\req{R25-F-O}{L'utente deve poter filtrare la lista per data iniziale di copertura}{\hyperref[uc:filtraggio-data-iniziale-copertura]{UC5.5}}
\req{R26-F-O}{L'utente deve poter filtrare la lista per data finale di copertura}{\hyperref[uc:filtraggio-data-finale-copertura]{UC5.6}}
\req{R27-F-O}{L'utente deve poter filtrare la lista per quantità ordinata}{\hyperref[uc:filtraggio-quantità-ordinata]{UC5.7}}
\req{R28-F-O}{L'utente deve poter filtrare la lista per prezzo totale del singolo ordine}{\hyperref[uc:filtraggio-prezzo-totale-ord]{UC5.8}}
\req{R29-F-O}{L'utente deve poter ordinare gli elementi della lista}{\hyperref[uc:ordinamento-elementi-lista]{UC6}}
\req{R30-F-O}{L'utente deve poter ordinare la lista rispetto al codice articolo}{\hyperref[uc:ordinamento-codice-articolo]{UC6.1}}
\req{R31-F-O}{L'utente deve poter ordinare la lista rispetto al codice fornitore}{\hyperref[uc:ordinamento-codice-fornitore]{UC6.2}}
\req{R32-F-O}{L'utente deve poter ordinare la lista rispetto alla data d'ordine}{\hyperref[uc:ordinamento-data-ordine]{UC6.3}}
\req{R33-F-O}{L'utente deve poter ordinare la lista in rispetto alla data iniziale di coppertura}{\hyperref[uc:ordinamento-data-iniziale-copertura]{UC6.4}}
\req{R34-F-O}{L'utente deve poter ordinare la lista in rispetto alla data finale di copertura}{\hyperref[uc:ordinamento-data-finale-copertura]{UC6.5}}
\req{R35-F-O}{L'utente deve poter ordinare la lista in rispetto alla quantità ordinata}{\hyperref[uc:ordinamento-quantità-ordinata]{UC6.6}}
\req{R36-F-O}{L'utente deve poter ordinare la lista in rispetto al prezzo totale del singolo ordine}{\hyperref[uc:ordinamento-prezzo-totale-ord]{UC6.7}}
\end{longtable}
\end{center}%

%tab qualità
\begin{center}
%\rowcolors{2}{lightest-grayest}{white}
\begin{longtable}{|p{2cm}|p{9cm}|p{2cm}|}
\caption{Tabella del tracciamento dei requisiti qualitativi}
\label{tab:requisiti-qualitativi}
\\ \hline
\centering \textbf{Requisito} & \centering \textbf{Descrizione} & \centering \textbf{Use Case} \arraybackslash \\
\hline 
\req{R37-Q-O}{Deve essere redatto un documento che descrive l'architettura del modulo}{-}
\req{R38-Q-O}{Deve essere redatto un documento che spieghi le scelte implementative effettuate}{-}
\req{R39-Q-O}{Il codice deve essere documentato tramite commenti}{-}
\req{R40-Q-D}{L'algoritmo finale scelto deve generare dei log di chiamata per manutenzioni future}{-}
\req{R41-Q-D}{L'algoritmo di ottimizzazione deve essere estensibile}{-}
\req{R42-Q-O}{I test devono coprire il 60\% del codice}{-}
\req{R43-Q-D}{L'algoritmo utilizza differenti tecniche di ottimizzazione}{-}
\req{R44-Q-F}{L'algoritmo utlizza il multithreading per cercare più soluzioni ammissibili}{-}
\end{longtable}
\end{center}%

%tab performance
\begin{center}
%\rowcolors{2}{lightest-grayest}{white}
\begin{longtable}{|p{2cm}|p{9cm}|p{2cm}|}
\caption{Tabella del tracciamento dei requisiti di performance}
\label{tab:requisiti-di-performance}
\\ \hline
\centering \textbf{Requisito} & \centering \textbf{Descrizione} & \centering \textbf{Use Case} \arraybackslash \\
\hline 
\req{R45-P-O}{L'algoritmo di ottimizzazione deve restituire un risultato entro 10 minuti dal tempo di lancio dello stesso}{-}
\end{longtable}
\end{center}%

%tab vincolo
\begin{center}
%\rowcolors{2}{lightest-grayest}{white}
\begin{longtable}{|p{2cm}|p{9cm}|p{2cm}|}
\caption{Tabella del tracciamento dei requisiti di vincolo}
\label{tab:requisiti-di-vincolo}
\\ \hline
\centering \textbf{Requisito} & \centering \textbf{Descrizione} & \centering \textbf{Use Case} \arraybackslash \\
\hline  
\req{R46-V-O}{La \textit{form} deve essere eseguita sull'ambiente di esecuzione \textit{.NET Framework}}{-}
\req{R47-V-O}{La \textit{form} e l'algoritmo devono essere codificate in $C\#$}{-}
\req{R48-V-O}{La versione utilizzata di $C\#$ deve essere $7.3$}{-}
\req{R49-V-O}{La versione utilizzata di \textit{.NET Framework} deve essere $4.8$}{-}
\req{R50-V-O}{L'algoritmo finale deve fonire una soluzione ammissibile}{-}
\end{longtable}
\end{center}%