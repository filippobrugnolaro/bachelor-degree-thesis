% !TEX encoding = UTF-8
% !TEX TS-program = pdflatex
% !TEX root = ../tesi.tex

%**************************************************************
\chapter{Introduzione}
\label{cap:introduzione}
%**************************************************************

\intro{Nel seguente capitolo si introduce brevemente l'azienda ospitante e il progetto affrontato.}\\

%Introduzione al contesto applicativo.\\

%\noindent Esempio di utilizzo di un termine nel glossario \\
%\gls{api}. \\

%\noindent Esempio di citazione in linea \\
%\cite{site:agile-manifesto}. \\

%\noindent Esempio di citazione nel pie' di pagina \\
%citazione\footcite{womak:lean-thinking} \\

%**************************************************************
\section{L'azienda}

\noindent{\myCompany}\footnote{Sito ufficiale: \url{https://www.ergon.it/}}(da qui in poi "\textit{Ergon}") è un'azienda italiana, fondata nel 1988, 
situata a Castelfranco Veneto.\\
Essa si occupa principalmente dello sviluppo di software \gls{erpg} per i settori dell'alimentare e dei trasporti, ma offre anche servizi di assistenza
che variano dalla sicurezza in ambito web alla vendita e installazione di software di terze parti.\\
L'azienda inoltre si è sviluppata in maniera costante negli anni e oggi può vantare una posizione di tutto rispetto tra le aziende dello stesso settore.
Attualmente fanno parte della stessa gestione:
\begin{itemize}
    \item \textit{Ergon Informatica S.R.L.}: che si occupa del software;
    \item \textit{Ergon S.R.L.}: che si occupa dei servizi tecnologici;
    \item \textit{Ergon Servizi S.R.L.}: che si occupa dei servizi amministrativi, logistici e di marketing delle altre due parti.
\end{itemize}
%Il logo dell'azienda è illustrato in figura.
%\includegraphics{}
Il prodotto proprietario e interamente sviluppato dall'azienda è \textit{ERGDIS}, sistema \gls{erpg}
il cui insieme dei moduli copre ogni aspetto della conduzione aziendale.\\
In particolare vengono gestiti vari aspetti che si dislocano dall'area amministrativa al controllo direzionale,
dall'area commerciale alla pianificazione e al controllo della produzione, dalla gestione acquisti alla logistica di magazzino, 
all'archiviazione ottica alla gestione della qualità.\\
Alcuni di essi, inoltre, si possono interfacciare con dispositivi automatici presenti in azienda, come, ad esempio,
linee di confezionamento o robot.\\
 

%**************************************************************
\section{L'idea}

Introduzione all'idea dello stage.

%**************************************************************
\section{Descrizione dello stage}

%**************************************************************
\subsection{Introduzione}

%**************************************************************
\subsection{Obiettivi}

%**************************************************************
\subsection{Analisi preventiva dei rischi}

%**************************************************************
\section{Organizzazione del testo}

\begin{description}
    \item[{\hyperref[cap:studio-fattibilita]{Il secondo capitolo}}] descrive ...
    
    \item[{\hyperref[cap:descrizione-stage]{Il terzo capitolo}}] approfondisce ...
    
    \item[{\hyperref[cap:analisi-requisiti]{Il quarto capitolo}}] approfondisce ...
    
    \item[{\hyperref[cap:progettazione-codifica]{Il quinto capitolo}}] approfondisce ...
    
    \item[{\hyperref[cap:verifica-validazione]{Il sesto capitolo}}] approfondisce ...
    
    \item[{\hyperref[cap:conclusioni]{Nel settimo capitolo}}] descrive ...
\end{description}

Riguardo la stesura del testo, relativamente al documento sono state adottate le seguenti convenzioni tipografiche:
\begin{itemize}
	\item gli acronimi, le abbreviazioni e i termini ambigui o di uso non comune menzionati vengono definiti nel glossario, situato alla fine del presente documento;
	\item per la prima occorrenza dei termini riportati nel glossario viene utilizzata la seguente nomenclatura: \emph{parola}\glsfirstoccur;
	\item i termini in lingua straniera o facenti parti del gergo tecnico sono evidenziati con il carattere \emph{corsivo}.
\end{itemize}