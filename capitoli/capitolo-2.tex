% !TEX encoding = UTF-8
% !TEX TS-program = pdflatex
% !TEX root = ../tesi.tex

%**************************************************************
\newgeometry{a4paper, left=30mm, right=30mm, top=5mm, bottom=30mm}
\chapter{Studio di fattibilità}
\label{cap:studio-fattibilita}
%**************************************************************

\noindent \intro{In questo capitolo viene esposto lo studio di fattibilità,
in cui verranno evidenziati i punti critici, i vantaggi e gli svantaggi delle
soluzioni analizzate}\\

%**************************************************************
\section{Introduzione allo studio}
\noindent Lo studio di fattibilità rappresenta una delle parti più corpose
del progetto in quanto viene richiesto un ampio periodo di autoapprendimento
dei principali algoritmi di \gls{ricercaoperativag} e di ottimizzazione combinatoria,
seguito da un'ulteriore approfondimento attraverso la consultazione di vari \textit{paper}\footnote[2]{Verranno citati nella bibliografia solo quelli ritenuti più significativi}
disponibili \textit{online}.\\

\noindent Questa prima parte definirà una prima scelta tra gli algoritmi disponibili in base alle informazioni reperite.\\

\noindent La seconda parte invece consiste nello sviluppo di micro-moduli di \textit{test} sulle scelte effettuate
precedentemente in modo tale da poter effettuare una analisi ed un confronto accurati basati su parametri che verranno
definiti successivamente.\\

\noindent Considerando anche il rapporto tra costi e risorse, le soluzioni che sono state identificate come le più plausibili e sottoposte a uno studio più approfondito sono:
\begin{enumerate}
    \item Algoritmo \textit{greedy}
    \item \textit{Tabu search}
    \item Algoritmo genetico
\end{enumerate}

\noindent Prima di proseguire con lo studio di fattibilità, è necessario dichiarare le metriche che sono state
utilizzate per effetuare un confronto equo tra gli algoritmi. Chiaramente esse derivano dagli obiettivi che
l'algoritmo deve soddisfare per risolvere il problema.\\
Vengono elencati in seguito i parametri:
\begin{itemize}
    \item \textbf{Efficienza}: la capacità dell'algoritmo di utilizzare meno risorse possibili per risolvere il problema.
    \item \textbf{Efficacia}: la capacità dell'algoritmo di risolvere il problema fornendo una soluzione il più possibile corretta.
    \item \textbf{Complessità implementativa}: quantitavo di risorse temporali impiegate per sviluppare l'algoritmo.
    \item \textbf{Paper}: dichiarazioni o dati di esperimenti già effettuati.
\end{itemize}

\newgeometry{a4paper, left=30mm, right=30mm, top=31mm, bottom=30mm}

%**************************************************************
\section{Soluzioni proposte}
\noindent Per ogni soluzione proposta viene effettuata una breve introduzione, seguita da vantaggi e svantaggi.
Gli pseudocodici che descrivono in maniera sintetica i micro-moduli di \textit{test} si basano su un problema di \textit{string replacement}
con \textit{input} di lunghezza uguale.\\
Si è volutamente scelto questo tipo di problema molto semplice poichè, se i micro-moduli fossero stati sviluppati intercalandoli all'interno del contesto,
si avrebbe avuto un enorme spreco di risorse temporali.

%**************************************************************
\subsection{Algoritmo \textit{Greedy}}
\noindent L'algoritmo \textit{greedy} ("\textit{goloso}") viene così chiamato poichè basa la ricerca di una
soluzione \gls{ammissibileg} ottima sulla scelta, secondo un criterio predefinto, della miglior soluzione disponibile ad ogni passo senza
rimettere in discussione la scelta appena effettuata.\\
Di seguito viene presentato lo pseudocodice dell'algoritmo \textit{greedy}.

\begin{algorithm}[!h]
    \captionsetup{labelformat=empty}
    \caption{Pseudocodice \textit{string replacement} - Algoritmo \textit{greedy}}
    \vspace{0.1cm}
    \hspace*{\algorithmicindent} \textbf{\textit{Input}:} {$stringa\_corretta$}, {$stringa\_errata$}\\
    \hspace*{\algorithmicindent} \textbf{\textit{Output}:} {$funzione\_obiettivo$}
    \begin{algorithmic}[1]
        \Procedure{My\_Greedy\_Algorithm}{$stringa\_corretta$, $stringa\_errata$}
        \State {$array_{str\_err} \gets generate\_array(stringa\_errata)$}
        \State {$array_{str\_corr} \gets generate\_array(stringa\_corretta)$}
        \State {$funzione\_obiettivo \gets array_{str\_corr}.Length$}
        \State {$pos \gets 0$}
        \ForAll {$element \in array_{str\_err}$}
            \If {$element \neq array_{str\_corr}[pos]$}
                \State $element = array_{str\_corr}[pos]$
            \EndIf
            \State {$funzione\_obiettivo \gets funzione\_obiettivo - 1$}
            \State {$pos \gets pos + 1$}
        \EndFor
        \State \Return {$funzione\_obiettivo$}
        \EndProcedure
    \end{algorithmic}
\end{algorithm}

\noindent \paragraph{Osservazioni}\hfill\\
Come si può notare, l'algoritmo \textit{greedy} è molto intuitivo e, in questo caso,
anche banale.
Infatti, ad ogni iterazione, definiti $x$ come l'elemento in posizione $pos$ nella
stringa errata e $y$ come l'elemento, nella stessa posizione, nella stringa corretta,
se $x \neq y$ si effettua un replace di $x$ con la scelta migliore disponibile
in quel momento, ovvero $y$ stesso.

\noindent \paragraph{Aspetti positivi}
\begin{itemize}
    \item Bassa complessità implementativa;
    \item Bassa complessità computazionale dell'algoritmo;
    \item Possibilità di effettuare scelte \textit{greedy} differenti in problemi di vaste dimensioni.
\end{itemize}
\noindent \paragraph{Aspetti negativi}
\begin{itemize}
    \item Possibilmente inefficace, può non raggiungere l'ottimo globale a causa
    delle scelte \textit{greedy} effettuate ad ogni iterazione che possono scartare soluzioni
    migliori nel lungo periodo.
\end{itemize}
%**************************************************************
\subsection{\textit{Tabu search}}
\noindent La \textit{Tabu search}\footnote[3]{Per una descrizione più accurata si legga la sezione \hyperref[sec:tabu-search]{4.4}} è un metodo, classificato come \textit{Trajectory Method}, basato su ricerca locale
in grado di eludere l'intrappolamento del metodo in un minimo locale sfruttando costantemente la memoria.\\
In particolare, oltre a ricordare la migliore soluzione corrente, vengono salvate, in quella che viene definita \textit{Tabu list},
anche le {$k$} mosse precedentemente effettuate in modo tale da non incombere nel rischio di un ciclo nel breve periodo.\\
Viene dunque orientata la ricerca tramite la modifica del vicinato in funzione della storia dell'esplorazione, ma anche tramite la diversificazione dei sottospazi di ricerca tramite il passaggio per soluzioni non ammissibili.\\
Di seguito viene presentato lo pseudocodice della \textit{Tabu search}.
\begin{algorithm}[!h]
    \captionsetup{labelformat=empty}
    \caption{Pseudocodice \textit{string replacement} - \textit{Tabu search}}
    \vspace{0.1cm}
    \hspace*{\algorithmicindent} \textbf{\textit{Input}:} {$stringa\_corretta$}, {$stringa\_errata$}, {$capienza\_tabu\_list$}, {$max\_iterazioni$}\\
    \hspace*{\algorithmicindent} \textbf{\textit{Output}:} {$funzione\_obiettivo$}
    \begin{algorithmic}[1]
        \Procedure{My\_Tabu\_Search}{$stringa\_corretta$, $stringa\_errata$, $capienza\_tabu\_list$, $max\_iterazioni$}
        \State {$sol_{curr} \gets stringa\_errata$}
        \State {$funzione\_obiettivo_{curr} \gets stringa\_corretta.Length$}
        \State {$funzione\_obiettivo_{best} \gets funzione\_obiettivo_{curr}$}
        \State {$conta \gets 0$}
        \While {$conta < max\_iterazioni$}
            \State {$vicinato \gets genera\_vicinato()$}
            \While{$vicinato.Length > 0$ \textbf{and} $conta < max\_iterazioni$}
                \State {$vicino_{migl} \gets ottieni\_miglior\_vicino(vicinato)$}
                \If {$vicino_{migl} \not\in tabu\_list$}
                    \State {$funzione\_obiettivo_{curr} \gets calculate\_fo(vicino_{migl})$}
                    \If {$funzione\_obiettivo_{curr} < funzione\_obiettivo_{best}$}
                        \State Inserisci {$vicino_{migl}$} nella {$tabu\_list$} considerando la {$capienza\_tabu\_list$}
                        \State $sol_{curr} \gets vicino_{migl}$
                        \State {$funzione\_obiettivo_{best} \gets funzione\_obiettivo_{curr}$}
                        \State {$conta \gets conta + 1$}
                        \State {$\textbf{break}$}
                    \EndIf
                \EndIf
                \State Rimuovi {$vicino_{migl}$} dal {$vicinato$}
                \State {$conta \gets conta + 1$}
            \EndWhile
        \EndWhile
        \State \Return {$funzione\_obiettivo$}
        \EndProcedure
    \end{algorithmic}
\end{algorithm}
\vspace*{\fill}
\noindent \paragraph{Osservazioni}\hfill\\
Notiamo come il ruolo della {$tabu\_list$} sia fondamentale in quanto, in caso di appartenza della mossa alla {$tabu\_list$}, non fa calcolare e
conseguentemente fare il confronto con {$funzione\_obiettivo_{best}$}. Inoltre se la {$funzione\_obiettivo_{curr}$} non è migliorante, l'iterazione
viene comunque contata.\\
Per quanto riguarda la condizione di {$\textbf{break}$}, essa viene messa poichè, avendo trovato una soluzione migliorante, non è più necessario
esplorare il vicinato corrente ed è dunque necessario crearne uno nuovo a partire dalla nuova soluzione.\\
\vspace*{\fill}
\newpage
\noindent \paragraph{Aspetti positivi}
\begin{itemize}
    \item Bassa complessità implementativa;
    \item Bassa complessità computazionale dell'algoritmo;
    \item Velocità di raggiungimento del minimo locale;
    \item Fuga da ottimi locali;
\end{itemize}

\noindent \paragraph{Aspetti negativi}
\begin{itemize}
    \item Possibilmente inefficace, può non raggiungere l'ottimo globale;
    \item Difficile calibrazione dei parametri;
    \item Numero di iterazioni necessario può essere molto alto;
    \item Necessità di una soluzione iniziale.
\end{itemize}
%**************************************************************
\subsection{\textit{Algoritmo genetico}}
\noindent L'algoritmo genetico è un metodo, classifcato come \textit{population based}, basato sul concetto che la natura
abbia la tendenza ad organizzarsi in strutture ottimizzate, in gran parte ispirato
alle teorie sull'evoluzione di \textit{Charles Darwin}\footnote[4]{Scienzato conosciuto
per le sue teorie sull'evoluzione secondo le quali gli individui di una
popolazione sono in competizione fra loro e, in questa
lotta per la sopravvivenza, l'ambiente opera una selezione naturale tramite la quale
vengono eliminati gli individui più deboli, cioè quelli meno adatti a sopravvivere a determinate
condizioni. Solo i più adatti sopravvivono e trasmettono i loro
caratteri ai figli.}.\\
\noindent In particolare, ad ogni iterazione, non viene mantenuta una sola soluzione, ma un'insieme di soluzioni, definita anche come popolazione.
Gli individui (soluzioni) vengono codificati tramite un cromosoma contenente una serie di geni (variabili decisionali del problema) e, per
ogni individuo appartente alla popolazione, viene associata quella che viene definita come la sua idoneità, tramite l'utilizzo di una
funzione di \textit{fitness}, che guida il processo di selezione, basato su metodi probabilistici (es: \textit{metodo Montecarlo}, \textit{linear ranking}, \textit{torneo-n}).\\
Prima di ogni iterazione vengono dunque presi gli individui e verranno accoppiati tramite degli operatori di ricombinazione
(es: \textit{crossover uniforme}, \textit{cut-point crossover}, \textit{mutazione}...)
per generare dei figli che assumeranno le migliori caratteristiche dei genitori. Alla fine dell'algoritmo verrà scelta la soluzione con la maggior \textit{fitness} possibile.\\

\begin{algorithm}[!h]
    \captionsetup{labelformat=empty}
    \caption{Pseudocodice \textit{string replacement} - \textit{Algoritmo genetico}}
    \vspace{0.1cm}
    \hspace*{\algorithmicindent} \textbf{\textit{Input}:} {$stringa\_corretta$}, {$crossover\_rate$}, {$mutation\_rate$}, \\\hspace*{54pt}{$max\_iterazioni$}\\
    \hspace*{\algorithmicindent} \textbf{\textit{Output}:} {$sol_{best}, fitness_{best}$}
    \begin{algorithmic}[1]
        \Procedure{My\_Genetic\_Algorithm}{$stringa\_corretta$, \par$stringa\_errata$, $crossover\_rate$, $mutation\_rate$, $max\_iterazioni$}
        \State {$arr_{str\_corr} \gets codifica(stringa\_corretta)$}
        \State {$lista\_pop \gets inizializza\_pop()$}
        \State {$set\_fitness()$}
        \State {$fitness_{best} \gets most\_fitness\_val(arr_{str\_corr})$}
        \State {$sol_{best} \gets most\_fitness\_sol(arr_{str\_corr})$}
        \State {$num_{crossover} \gets calculate\_num\_crossover(crossover\_rate$, $lista\_pop)$}
        \State {$conta \gets 0$}
        \algstore{part 1}
    \end{algorithmic}
\end{algorithm}
\begin{algorithm}
    \begin{algorithmic}
        \algrestore{part 1}
        \While {$conta < max\_iterazioni$}
        \State {$lista\_pop \gets seleziona\_individui(num_{crossover})$}
        \State {$set\_fitness()$}
            \State {$i \gets 0$}
            \While {$i < num_{crossover}$}
                \State {$lista\_genitori \gets seleziona\_individui(2)$}
                \State {$figlio \gets genera\_figlio(lista\_genitori)$}
                \State Aggiungi il {$figlio$} in coda alla lista {$nuova\_pop$}
                \State {$i \gets i + 1$}
            \EndWhile
            \State {$lista\_pop \gets nuova\_pop$}
            \State {$lista\_pop \gets mutazione\_rand(lista\_pop)$}
            \State {$set\_fitness()$}
            \State {$fitness_{curr} \gets most\_fitness\_val(arr_{str\_corr})$}
            \State {$sol_{curr} \gets most\_fitness\_sol(arr_{str\_corr})$}
            \If {$fitness_{curr} > fitness_{best}$}
                \State {$fitness_{best} \gets fitness_{curr}$}
                \State {$sol_{best} \gets sol_{curr}$}
            \EndIf
            \State {$conta \gets conta + 1$}
        \EndWhile
        \State \Return {$sol_{best}$, $fitness_{best}$}
        \EndProcedure
    \end{algorithmic}
\end{algorithm}

\noindent \paragraph{Osservazioni}\hfill\\
Si noti come, per ogni popolazione, vengano scelti un numero di individui dipendente dal {$crossover\_rate$}
e vengano effettuati {$num_{crossover}$} \textit{crossover} scegliendo 2 individui dalla popolazione che fungono da genitori.
Viene dunque creata una nuova popolazione di figli a cui viene anche applicata una mutazione a un figlio \textit{random} per
variare il patrimonio genetico, in modo tale da avere più possibilità di trovare buone caratteristiche.

\noindent \paragraph{Aspetti positivi}
\begin{itemize}
    \item Fuga da ottimi locali;
    \item Analisi di più sottospazi delle soluzioni, grazie alla varietà della popolazione;
    \item Utilizzo modelli probabilistici.
\end{itemize}

\noindent \paragraph{Aspetti negativi}
\begin{itemize}
    \item Modesta complessità implementativa;
    \item Possibilmente inefficace, può non raggiungere l'ottimo globale;
    \item Difficile calibrazione dei parametri;
    \item Difficile codifica degli individui in alcuni problemi;
    \item Numero di iterazioni necessario può essere molto alto.
\end{itemize}
\newpage
%**************************************************************
\section{Conclusioni dello studio}
\label{conclusione-studio-fattibilita}
\noindent Lo studio è servito per capire come funzionassero gli algoritmi e quali fossero i loro pregi e difetti.
Di seguito si hanno i risultati dell'analisi dei 3 algoritmi eseguiti singolarmente.\\

\renewcommand{\arraystretch}{1.6}

% tabella con i risultati
\begin{center}
    \begin{longtable}{|m{3cm}|m{3cm}|m{3cm}|m{3cm}|}
    \caption{Tabella dei risultati dell'analisi degli algoritmi dopo 10 esecuzioni}
    \label{tab:risultati-studio-fattibilita}
    \\ \hline
    \centering \textbf{Tipologia} & \centering \textbf{Efficacia (\%)} & \centering \textbf{Efficienza (ms)} & \centering \textbf{Tempo di realizzazione (h)} \arraybackslash \\
    \hline
    \centering Algoritmo \textit{greedy} & \centering 100 & \centering 0,45 & \centering 0,5 \arraybackslash \\
    \hline
    \centering \textit{Tabu search} & \centering 98 & \centering 1289,76 & \centering 2,5 \arraybackslash \\
    \hline
    \centering Algortimo Genetico & \centering 92 & \centering 5127,24 & \centering 5 \arraybackslash \\
    \hline
    \end{longtable}
\end{center}%

\noindent L'algoritmo \textit{greedy} risulta il più efficiente ed efficace in questo specifico caso,
ma per il problema che si andrà a risolvere non può essere considerato
molto soddisfacente in quanto le scelte
non vengono mai rimesse in discussione e potrebbero dunque escludere soluzioni potenzialmente migliori
(esempio \hyperref[cap:introduzione]{Introduzione}).\\
Per quanto riguarda l'algoritmo genetico, sebbene fosse interessante per il mantenimento di più soluzioni ad ogni iterazione, si è dimostrato
molto più lento rispetto ai due precedenti. Inoltre una maggiore complessità realizzativa lo ha portato all'esclusione.\\
Si è optato quindi per l'utilizzo della \textit{Tabu search} in quanto rappresentava un buon compromesso sia a livello
di efficacia ed efficienza che a livello implementativo. Per quanto riguarda la soluzione iniziale,
questa può essere generata a partire da un algoritmo \textit{greedy}, data la sua bassa complessità computazionale.
In questo modo è possibile avere già a disposizione una buona soluzione di base.\\
Tuttavia, sebbene la \textit{Tabu search} possieda una sola soluzione ad ogni iterazione,
è stato pensato che l'algoritmo possa essere lanciato in più \textit{thread},
riuscendo dunque in un certo senso a simulare il mantenimento di più soluzioni
dell'algoritmo genetico, passando ai \textit{thread} chiaramente soluzioni iniziali differenti.