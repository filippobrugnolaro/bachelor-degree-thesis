% !TEX encoding = UTF-8
% !TEX TS-program = pdflatex
% !TEX root = ../tesi.tex

\newgeometry{a4paper, left=30mm, right=30mm, top=5mm, bottom=30mm}
%**************************************************************
\chapter{Conclusioni}
\label{cap:conclusioni}
%**************************************************************
\noindent \intro{In questo ultimo capitolo viene effettuata una analisi retrospettiva sullo \textit{stage}
focalizzandosi sul raggiungimento degli obiettivi, sulle conoscenze acquisite e sulla valutazione personale del percorso.}
%**************************************************************
\section{Prodotto finale}
\noindent Come descritto nei capitoli precedenti, il prodotto finale consiste in una \textit{\gls{windowsformg}} che
ottimizza l'insieme degli ordini che vengono effettuati per soddisfare i
fabbisogni di un dato periodo.\\

\noindent Dopo aver inserito tutti gli \textit{input} necessari,
viene determinata una soluzione iniziale tramite un algoritmo \textit{greedy} e poi, attraverso la
\textit{Tabu search}, si crea una nuova soluzione che viene visualizzata in una lista
filtrabile contenente tutte le informazioni di ogni ordine.

\section{Consuntivo delle tempistiche}
\noindent Nella Tabella \ref{tab:attivita-ore-fine} viene esposto il consuntivo orario finale.
\renewcommand{\arraystretch}{1.6}

% tabella con i risultati
\begin{center}
    \begin{longtable}{m{9cm}m{3cm}}
    \caption{Tabella delle attività con le corrispettive ore consuntivate}
    \label{tab:attivita-ore-fine}
    \\ \hline
    \centering \textbf{Descrizione attività} & \centering \textbf{Ore consuntivate} \arraybackslash \\
    \hline
    \centering Analisi del modulo \textit{software}
    esistente e delle funzionalità da realizzare & \centering 24 \arraybackslash \\
    \hline
    \centering Studio di fattibilità e Studio di algoritmi e tecniche di
    \gls{ricercaoperativag}
    e Ottimizzazione Combinatoria & \centering 100 \arraybackslash \\
    \hline
    \centering Studio delle tecnologie aziendali necessarie allo sviluppo del
    modulo & \centering 20 (-12) \arraybackslash \\
    \hline
    \centering Sviluppo di micro-moduli di \textit{test} per gli algoritmi
    studiati & \centering 8 \arraybackslash \\
    \hline
    \centering Sviluppo modulo effettivo, generazione/lettura dei vincoli
    e parametrizzazione tramite pesi delle variabili & \centering 96 (+4) \arraybackslash \\
    \hline
    \centering \textit{Test} e Validazione & \centering 28 (+8) \arraybackslash \\
    \hline
    \centering Stesura della documentazione
    del prodotto sviluppato & \centering 24 \arraybackslash \\
    \hline
    \end{longtable}
\end{center}%

\noindent È evidente come, rispetto al preventivo nella Tabella
\ref{tab:attivita-ore-inizio} (Sezione §\ref{sec:pianificazione-lavoro}),
la gestione delle ore sia stata abbastanza in linea con quanto dichiarato.
Tuttavia, si può notare come le ore risparmiate per lo studio delle tecnologie
siano state impiegate in una minima parte nello sviluppo del modulo effettivo e
la restante parte per il \textit{testing} (vedasi problema del \textit{\gls{databaseg}} nella
Sezione §\ref{sec:risultati-test}).
Ciò si è verificato perchè le conoscenze pregresse (come spiegato nella Sezione §\ref{sec:conoscenze-acquisite})
hanno permesso un'apprendimento più rapido del previsto.

\newgeometry{a4paper, left=30mm, right=30mm, top=31mm, bottom=30mm}
%**************************************************************
\section{Soddisfacimento dei requisiti}
\noindent Come descritto nella Sezione §\ref{sec:validazione-requisiti}, il prodotto
soddisfa la maggior parte delle necessità descritte nell'analisi dei requisiti.
In particolare sono stati realizzati tutti i requisiti funzionali, la maggior parte
di quelli qualitativi, tutti quelli di performance e di vincolo.\\
Nella Tabella \ref{tab:requisiti-riepilogo-validazione} viene fornita una chiara visuale
del soddisfacimento dei requisiti.
\begin{center}
    \rowcolors{2}{lightest-grayest}{white}
    \begin{longtable}{|p{2.5cm}|p{2.5cm}|p{2.5cm}|p{2.5cm}|}
    \caption{Riepilogo della validazione dei requisiti}
    \label{tab:requisiti-riepilogo-validazione}
    \\ \hline
    \rowcolor{lighter-grayer}
    \centering \textbf{Tipo} & \centering \textbf{Obbligatori} & \centering \textbf{Desiderabili} & \centering \textbf{Facoltativi}\arraybackslash \\
    \hline
    \reqsum{Soddisfatti}{45}{2}{0}
    \reqsum{Non Soddisfatti}{0}{2}{1}
    \end{longtable}
\end{center}%

%**************************************************************
\section{Raggiungimento degli obiettivi}
Nella Tabella \ref{tab:raggiungimento-obiettivi} viene illustrato il riepilogo del soddisfacimento degli obiettivi
\renewcommand{\arraystretch}{1.55}

% tabella con i risultati
\begin{center}
    \begin{longtable}{m{3cm}m{9cm}m{2cm}}
    \caption{Riepologo del soddisfacimento degli obiettivi}
    \label{tab:raggiungimento-obiettivi}
    \\ \hline
    \centering \textbf{Codice Obiettivo} & \centering \textbf{Descrizione Obiettivo} & \centering \textbf{Risultato} \arraybackslash \\
    \hline
    \centering OB1 & Sviluppo programmi per inserimento
    dei vincoli che delimitano il problema & \centering Soddisfatto \arraybackslash \\
    \hline
    \centering OB2 & Redazione di un documento che riporti
    i risultati ottenuti nello studio di fattibilità & \centering Soddisfatto \arraybackslash \\
    \hline
    \centering OB3 & Sviluppo di micro-moduli di prototipizzazione degli algoritmi analizzati & \centering Soddisfatto \arraybackslash \\
    \hline
    \centering OB4 & Sviluppo modulo \textit{software} per
    la soluzione del problema & \centering Soddisfatto \arraybackslash \\
    \hline
    \centering OB5 & Acquisizione di competenze sull’utilizzo
    di algoritmi di \gls{ricercaoperativag} e
    applicazione in un caso reale & \centering Soddisfatto \arraybackslash \\
    \hline
    \centering DE1 & Utilizzo di più tecniche e combinazione dei risultati
    ottenuti o individuazione della miglior soluzione attraverso
    opportuni \textit{KPI} (\textit{Key Performance Indicator}) & \centering Non Soddisfatto \arraybackslash \\
    \hline
    \centering FA1 & Utilizzo del \textit{multithreading}
    nelle fasi in cui è richiesta una
    maggiore capacità di calcolo & \centering Non Soddisfatto \arraybackslash \\
    \hline
    \end{longtable}
\end{center}%

\noindent Come si può notare gli obiettivi obbligatori
sono stati tutti soddisfatti. Questi, infatti, rappresentavano
il minimo indispensabile per il superamento dell'attività di \textit{stage}.\\
Gli obiettivi desiderabili e facoltativi non sono stati soddisfatti.
Il motivo sta nel fatto che non c'è stato il tempo necessario per svolgerli.
Si è deciso, dunque, di dedicare le ore rimanenti
al miglioramento della documentazione già esistente, che sarebbe risultata
più utile rispetto a iniziare un obiettivo che sicuramente non si sarebbe portato a termine.

%**************************************************************
\section{Conoscenze acquisite}
\label{sec:conoscenze-acquisite}
\noindent Per la realizzazione di questo progetto, sono state fondamentali
molte nozioni apprese durante il corso di studi, dai linguaggi ai paradigmi.
Inoltre, aver appreso in precedenza come modellare e risolvere un problema di ottimizzazione tramite la \gls{ricercaoperativag},
ha sicuramente aiutato nell'analisi del problema, che altrimenti sarebbe stata certamente più complicata.\\
Lo \textit{stage} ha permesso però anche di incrementare notevolmente il mio bagaglio
sia tecnico che personale.
Vengono ora elencate le principali conoscenze acquisite.
\paragraph{\textit{Informix}, \textit{C\#} e \textit{MSTest}}\hfill\\\\
\noindent Per lo sviluppo del progetto mi sono servito principalmente di queste tecnologie:
\begin{itemize}
    \item \textit{Informix} per \textit{\gls{databaseg}};
    \item \textit{C\#} come linguaggio di programmazione;
    \item \textit{MSTest} come \textit{framework} di \textit{test}.
\end{itemize}
\noindent Non avendole mai affrontate né all’università né in privato, ho dovuto
cominciare praticamente da zero, ma poichè struttura, regole e utilizzo sono simili
a tecnologie conosciute, come ad esempio \textit{MySQL} per \textit{Informix}, \textit{\gls{C++g}} e \textit{\gls{javag}} per \textit{C\#},
e \textit{\gls{JUnitg}} $^{[g]}$ per \textit{MSTest}, l’apprendimento è stato molto semplificato.
Ciò mi ha permesso di acquisire le conoscenze necessarie in tempi rapidissimi
e grazie anche all'utilizzo di un approccio conosciuto come \textit{learning
by doing}.
\paragraph{Analisi e modellazione del problema}\hfill\\\\
Analizzare e modellare un problema è una delle competenze che si devono apprendere
in un corso di laurea in informatica. Tuttavia i problemi con cui si ha a che fare
nel mondo accademico sono molto spesso \textit{standard} e risolvibili tramite tecniche
ben conosciute.\\
Questo progetto mi ha posto davanti per la prima volta un problema
non banale da risolvere da solo. Infatti uno studio preliminare forse poco accurato
mi ha portato a effettuare per buona parte dello \textit{stage} molte domande al \textit{tutor}.
È chiaro dunque come sia fondamentale capire il problema che si ha di fronte nella
sua interezza prima di iniziare la progettazione e lo sviluppo.

\paragraph{Gestione delle risorse}\hfill\\\\
Il mondo universitario pone lo studente a organizzare le sue risorse temporali disponibili
per preparare al meglio gli esami. Tutto ciò, nella mia esperienza, è diverso con la
gestione delle risorse in ambito aziendale. L'organizzazione diventa fondamentale
e la schedulazione degli eventi e degli obiettivi diventa imprescindibile.
Lo svolgimento dello \textit{stage} mi ha permesso dunque di crescere molto su questo aspetto
che avevo già rafforzato nel corso di \textit{Ingegneria del software}.

%**************************************************************
\section{Valutazione complessiva}
\noindent Valuto lo \textit{stage} presso \textit{Ergon Informatica} in maniera molto positiva.\\
I due mesi trascorsi in azienda sono risultati molto leggeri e piacevoli soprattutto grazie
ad un ambiente di lavoro molto motivante e, a tratti, anche divertente.\\

\noindent Grazie ad un ottimo rapporto con il \textit{\textit{tutor}} aziendale, non ho mai dovuto
affrontare difficoltà bloccanti in autonomia. Gianluca, anche durante il suo periodo di vacanza, è sempre stato disponibile
ad ascoltare dubbi e perplessità per fornirmi indicazioni utili che mi guidassero
verso la via corretta e mi permettessero di continuare il lavoro.\\
Un altro aspetto che ho molto apprezzato è stato l'interesse di alcuni colleghi
verso il progetto che stavo realizzando, il che mi faceva sentire parte integrante
dell'azienda.\\

\noindent Per quanto concerne la parte di realizzazione del progetto, tutto è proseguito secondo quanto programmato.\\
Personalmente ho trovato più difficile la parte di inquadramento del problema e di scelta della funzione di valutazione.
Quest'ultima è stata particolarmente complessa perchè ha portato a
effettuare delle prove empiriche e aggiustare la funzione tentativo dopo tentativo.
Ho avuto modo di affrontare tematiche nuove legate alla risoluzione di problemi
relativi al mondo reale e questo mi ha permesso di rimanere costantemente motivato.\\

\noindent Concludendo, questo progetto di \textit{stage} è stato per me
molto soddisfacente per modalità di svolgimento, per il prodotto finale, per le conoscenze acquisite e per le persone conosciute.