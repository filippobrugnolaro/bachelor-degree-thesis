% !TEX encoding = UTF-8
% !TEX TS-program = pdflatex
% !TEX root = ../tesi.tex

\newgeometry{a4paper, left=30mm, right=30mm, top=5mm, bottom=30mm}
%**************************************************************
\chapter{Conclusioni}
\label{cap:conclusioni}
%**************************************************************
\noindent \intro{In questo ultimo capitolo viene effettuata una analisi retrospettiva sullo stage
focalizzandosi sul raggiungimento degli obiettivi, sulle conoscenze acquisite e sulla valutazione personale del percorso.}
%**************************************************************
\section{Prodotto finale}
\noindent Come descritto nei capitoli precedenti, il prodotto finale consiste in una WinForm che
ottimizza l'insieme degli ordini che vengono effettuati per soddisfare i
fabbisogni di un dato periodo.\\

\noindent Dopo aver inserito tutti gli input necessari,
viene determinata una soluzione iniziale tramite un algoritmo greedy e poi attraverso la
tabu search viene creata una nuova soluzione che viene visualizzata in una lista
filtrabile contente tutte le informazioni di ogni ordine.
%**************************************************************
\section{Raggiungimento degli obiettivi}
\noindent Come descritto nella sezione §\ref{sec:validazione-requisiti} il prodotto
soddisfa la maggior parte delle necessità descritte nell'analisi dei requisiti.
In particolare sono stati realizzati tutti i requisiti funzionali, la maggior
di quelli qualitativi, tutti i quelli di performance e tutti quelli di vincolo.\\
Nella tabella \ref{tab:requisiti-riepilogo-validazione} viene fornita una chiara visuale
del soddisfacimento dei requisiti.
\begin{center}
    \rowcolors{2}{lightest-grayest}{white}
    \begin{longtable}{|p{2.5cm}|p{2.5cm}|p{2.5cm}|p{2.5cm}|}
    \caption{Riepilogo della validazione dei requisiti}
    \label{tab:requisiti-riepilogo-validazione}
    \\ \hline
    \rowcolor{lighter-grayer}
    \centering \textbf{Tipo} & \centering \textbf{Obbligatori} & \centering \textbf{Desiderabili} & \centering \textbf{Facoltativi}\arraybackslash \\
    \hline
    \reqsum{Soddisfatti}{45}{2}{0}
    \reqsum{Non Soddisfatti}{0}{2}{1}
    \end{longtable}
\end{center}%
%**************************************************************
\section{Conoscenze acquisite}
\noindent Per la realizzazione di questo progetto sono state fondamentali
molte nozioni apprese durante il corso di studi, dai linguaggi ai paradigmi.\\
Lo \textit{stage} ha permesso però anche di incrementare notevolmente il mio bagaglio
sia tecnico che personale.
Vengono ora elencate le principali conoscenze acquisite.
\paragraph{Informix, C\# e MSTest}\hfill\\\\
\noindent Per lo sviluppo del progetto mi sono servito principalmente di queste tecnologie:
\begin{itemize}
    \item Informix per database
    \item C\# come linguaggio di programmazione
    \item MSTest come framework di test
\end{itemize}
\newgeometry{a4paper, left=30mm, right=30mm, top=31mm, bottom=30mm}
\noindent Non avendole mai affrontate né all’università né in privato ho dovuto
cominciare praticamente da zero, ma poichè struttura, regole e utilizzo sono simili
a tecnologie conosciute, come ad esempio MySql per Informix, C++ e Java per C\#,
e JUnit per MSTest, l’apprendimento è stato molto semplificato.
Ciò mi ha permesso di acquisire le conoscenze necessarie in tempi rapidissimi
e grazie anche all'utilizzo di un approccio conosciuto come Learning
by doing.
\paragraph{Analisi e modellazione del problema}\hfill\\\\
Analizzare e modellare un problema è una delle competenze che si devono apprendere
in un corso di laurea in informatica. Tuttavia i problemi con cui si ha a che fare
nel mondo accademico sono molto spesso standard e risolvibili tramite tecniche
ben conosciute.\\
Questo progetto mi ha posto davanti per la prima volta un problema
non banale da risolvere da solo. Infatti uno studio preliminare forse poco accurato
mi ha portato a effettuare molte domande al tutor durante lo stage.
É chiaro dunque come sia fondamentale capire il problema che si ha di fronte nella
sua completa interezza prima di iniziare la progettazione e lo sviluppo.

\paragraph{Gestione delle risorse}\hfill\\\\
Il mondo universitario pone lo studente a organizzare le sue risorse temporali disponibili
per preparare al meglio gli esami. Tutto ciò non ha nulla a che vedere con la
gestione delle risorse in ambito aziendale. L'organizzazione diventa fondamentale
e la schedulazione degli eventi e degli obiettivi diventa imprescindibile.
Lo svolgimento dello stage mi ha permesso dunque di crescere molto su questo aspetto
che avevo già rafforzato nel corso di Ingegneria del software.

%**************************************************************
\section{Valutazione complessiva}
\noindent Lo \textit{stage} presso Ergon Informatica lo valuto in maniera molto positiva.\\
I due mesi trascorsi in azienda sono risultati molto leggeri e piacevoli soprattutto grazie
ad un ambiente di lavoro molto motivante e, a tratti, anche divertente.\\

\noindent Grazie ad un ottimo rapporto con il \textit{tutor} aziendale, non ho mai dovuto
affrontare difficoltà bloccanti in autonomia. Gianluca è sempre stato disponibile
ad ascoltare dubbi e perplessità per fornirmi indicazioni utili che mi guidassero
verso la via corretta e mi permettessero di continuare il lavoro.\\
Un altro aspetto che ho molto apprezzato è stato l'interesse di alcuni colleghi
verso il progetto che stavo realizzando, il che mi faceva sentire parte integrante
dell'azienda.\\

\noindent Per quanto concerne la parte di realizzazione del progetto, tutto è proseguito secondo quanto programmato.\\
Personalmente ho trovato più difficile la parte di inquadramento del problema e di scelta della funzione di valutazione.
Quest'ultima è stata particolarmente complessa perchè non esisteva una qualche procedura
che descrivesse come arrivare a definire la funzione di valutazione, ma bisognava
effettuare delle prove empiriche e aggiustare la funzione tentativo dopo tentativo.
Ho avuto modo di affrontare tematiche nuove legate alla risoluzione di problemi
legati al mondo reale e questo mi ha permesso di rimanere costantemente motivato.\\

\noindent Concludendo, questo progetto di \textit{stage} è stato per me
molto soddisfacente per modalità di svolgimento, per il prodotto finale, per le conoscenze acquisite e per le persone conosciute.