% !TEX encoding = UTF-8
% !TEX TS-program = pdflatex
% !TEX root = ../tesi.tex

\newgeometry{a4paper, left=30mm, right=30mm, top=5mm, bottom=30mm}
%**************************************************************
\chapter{Conclusioni}
\label{cap:conclusioni}
%**************************************************************
\noindent \intro{In questo ultimo capitolo viene effettuata una analisi retrospettiva sullo stage
focalizzandosi sul raggiungimento degli obiettivi, sulle conoscenze acquisite e sulla valutazione personale del percorso.}
%**************************************************************
\section{Prodotto finale}
\noindent Come descritto in precedenza il prodotto finale realizza 
\newgeometry{a4paper, left=30mm, right=30mm, top=31mm, bottom=30mm}
%**************************************************************
\section{Raggiungimento degli obiettivi}

\begin{center}
    \rowcolors{2}{lightest-grayest}{white}
    \begin{longtable}{|p{2.5cm}|p{2.5cm}|p{2.5cm}|p{2.5cm}|}
    \caption{Riepilogo della validazione dei requisiti}
    \label{tab:requisiti-riepilogo-validazione}
    \\ \hline
    \rowcolor{lighter-grayer}
    \centering \textbf{Tipo} & \centering \textbf{Obbligatori} & \centering \textbf{Desiderabili} & \centering \textbf{Facoltativi}\arraybackslash \\
    \hline
    \reqsum{Soddisfatti}{45}{2}{0}
    \reqsum{Non Soddisfatti}{0}{2}{1}
    \end{longtable}
\end{center}%

%**************************************************************
\section{Conoscenze acquisite}

%**************************************************************
\section{Valutazione complessiva}
