% !TEX encoding = UTF-8
% !TEX TS-program = pdflatex
% !TEX root = ../tesi.tex

%**************************************************************
% Sommario
%**************************************************************
\cleardoublepage
\phantomsection
\pdfbookmark{Sommario}{Sommario}
\begingroup
\let\clearpage\relax
\let\cleardoublepage\relax
\let\cleardoublepage\relax

\chapter*{Sommario}

\noindent Il presente documento descrive il
lavoro svolto durante il periodo di \textit{stage},
della durata di circa 320 ore, dal laureando
Brugnolaro Filippo presso l'azienda
\textit{\myCompany} dall'11/07/2022 al
05/09/2022.\\
\noindent Lo \textit{stage} consiste nella progettazione e nello sviluppo di un modulo \textit{software} volto
ad assistere un'azienda nella fase di approvvigionamento dei prodotti dai propri fornitori, supportandola nello scegliere
da quale fornitore e quando acquistare i prodotti.\\
Nei capitoli verrà effettuata una introduzione al problema, illustrato lo studio di fattibilità e l'analisi dei requisiti.
Successivamente verranno presentate la progettazione e codifica, seguite dalle loro fasi di verifica e validazione.
Infine verrano esposte alcune considerazioni finali e di carattere personale.\\

\noindent Riguardo la stesura del testo, relativamente al documento, sono state adottate le seguenti convenzioni tipografiche:
\begin{itemize}
	\item i termini ambigui o di uso non comune menzionati vengono definiti nel glossario alla fine del presente documento;
	\item per la prima occorrenza dei termini riportati nel glossario viene utilizzata la seguente nomenclatura: \emph{parola}\glsfirstoccur;
	\item i termini in lingua straniera o facenti parti del gergo tecnico sono evidenziati con il carattere \emph{corsivo}.
\end{itemize}

%\vfill
%
%\selectlanguage{english}
%\pdfbookmark{Abstract}{Abstract}
%\chapter*{Abstract}
%
%\selectlanguage{italian}

\endgroup			

\vfill

