%**************************************************************
% file contenente le impostazioni della tesi
%**************************************************************

%**************************************************************
% Frontespizio
%**************************************************************

% Autore
\newcommand{\myName}{Filippo Brugnolaro}

% Matricola
\newcommand{\myFreshman}{1217321}

\newcommand{\myTitle}{Sviluppo di un modulo software per la gestione degli ordini di acquisto con l'utilizzo di metodi euristici di ottimizzazione}

% Tipo di tesi                   
\newcommand{\myDegree}{Tesi di laurea}

% Università             
\newcommand{\myUni}{Università degli Studi di Padova}

% Facoltà       
\newcommand{\myFaculty}{Corso di Laurea in Informatica}

% Dipartimento
\newcommand{\myDepartment}{Dipartimento di Matematica "Tullio Levi-Civita"}

% Titolo del relatore
\newcommand{\profTitle}{Prof.}

% Relatore
\newcommand{\myProf}{Luigi De Giovanni}

% Luogo
\newcommand{\myLocation}{Padova}

% Anno accademico
\newcommand{\myAA}{2021-2022}

% Data discussione
\newcommand{\myTime}{Settembre 2022}

% Azienda
\newcommand{\myCompany}{Ergon Informatica srl}


%**************************************************************
% Impostazioni di impaginazione
% see: http://wwwcdf.pd.infn.it/AppuntiLinux/a2547.htm
%**************************************************************

\setlength{\parindent}{14pt}   % larghezza rientro della prima riga
\setlength{\parskip}{0pt}   % distanza tra i paragrafi


%**************************************************************
% Impostazioni di biblatex
%**************************************************************
\bibliography{bibliografia} % database di biblatex 

\defbibheading{bibliography} {
    \cleardoublepage
    \phantomsection
    \addcontentsline{toc}{chapter}{\bibname}
    \chapter*{\bibname\markboth{\bibname}{\bibname}}
}

\setlength\bibitemsep{1.5\itemsep} % spazio tra entry

\DeclareBibliographyCategory{opere}
\DeclareBibliographyCategory{web}

\addtocategory{opere}{womak:lean-thinking}
\addtocategory{web}{site:agile-manifesto}

\defbibheading{opere}{\section*{Riferimenti bibliografici}}
\defbibheading{web}{\section*{Siti Web consultati}}


%**************************************************************
% Impostazioni di caption
%**************************************************************
\captionsetup{
    tableposition=top,
    figureposition=bottom,
    font=small,
    format=hang,
    labelfont=bf
}

%**************************************************************
% Impostazioni di glossaries
%**************************************************************
\makeglossaries

%**************************************************************
% Acronimi
%**************************************************************
%\newacronym[description={\glslink{apig}{Application Program Interface}}]
%    {api}{API}{Application Program Interface}
%
%\newacronym[description={\glslink{umlg}{Unified Modeling Language}}]
%    {uml}{UML}{Unified Modeling Language}

%**************************************************************
% Glossario
%**************************************************************
\newglossaryentry{apig}
{
    name=\glslink{api}{API},
    text=API,
    sort=api,
    description={(\emph{Application Programming Interface})
    Intermediario \textit{software} che permette a due
    applicazioni non correlate di comunicare tra loro}
}

\newglossaryentry{umlg}
{
    name=\glslink{uml}{UML},
    text=UML,
    sort=uml,
    description={(\emph{Unified Model Language})
    \textit{Standard} utilizzato nell'ingegneria del \textit{software}
    per descrivere un sistema informatico, consentendo ai
    vari ruoli (sviluppatore, tester, analista,ecc.) di
    comunicare tramite lo stesso linguaggio. Si avvale di
    diversi tipi di diagrammi, statici e dinamici,
    per fornire diverse viste di uno stesso sistema \cite{site:def-uml}}
}

\newglossaryentry{erpg}
{
    name=\glslink{erp}{ERP},
    text=ERP,
    sort=erp,
    description={(\emph{Enterprise Resource Planning})
    Tipologia di \textit{software} che integra tutti i processi
    di business rilevanti di un'azienda e tutte le funzioni
    aziendali, ad esempio vendite, acquisti, gestione magazzino,
    finanza, contabilità etc. \cite{site:wiki}}
}

\newglossaryentry{.netg}
{
    name=\glslink{.netframework}{.NET Framework},
    text=.NET Framework,
    sort=.netframework,
    description={Ambiente di esecuzione runtime
    della piattaforma tecnologica \textit{.NET} in cui
    vengono gestite le applicazioni destinate allo stesso
    \textit{.NET Framework} \cite{site:wiki}}
}

\newglossaryentry{devexpressg}
{
    name=\glslink{devexpress}{DevExpress},
    text=DevExpress,
    sort=devexpress,
    description={\textit{Framework} utile per lo sviluppo
    di applicazioni desktop \cite{site:devexpress-docs}}
}

\newglossaryentry{serverconsolidationg}
{
    name=\glslink{serverconsolidation}{Server Consolidation},
    text=server consolidation,
    sort=serverconsolidation,
    description={Approccio all'utilizzo efficiente
    delle risorse dei \textit{server} dei computer al fine di
    ridurre il numero totale di \textit{server} o posizioni di
    \textit{server} richiesti da un'organizzazione \cite{site:def-server-cons}}
}

\newglossaryentry{ergdisg}
{
    name=\glslink{ergdis}{ERGDIS},
    text=ERGDIS,
    sort=ergdis,
    description={\textit{Software} proprietario dell'azienda \textit{\myCompany}}
}

\newglossaryentry{stakeholdersg}
{
    name=\glslink{stakeholder}{Stakeholders},
    text=stakeholders,
    sort=stakeholders,
    description={Coloro che a vario titolo hanno influenza su un
    prodotto o su un progetto}
}

\newglossaryentry{windowsformg}
{
    name=\glslink{windowsform}{Windows form},
    text=Windows form,
    sort=windowsform,
    description={Applicazione basata su eventi
    supportata da \textit{.NET Framework} di \textit{Microsoft} \cite{site:wiki}}
}

\newglossaryentry{ricercaoperativag}
{
    name=\glslink{ricercaoperativa}{Ricerca Operativa},
    text=Ricerca Operativa,
    sort=ricercaoperativa,
    description={Scienza che fornisce strumenti matematici e algoritmici di
    supporto alle attività decisionali in cui occorre gestire
    e coordinare attività e risorse limitate \cite{site:wiki}}
}

\newglossaryentry{ammissibileg}
{
    name=\glslink{ammissibile}{Soluzione Ammissibile},
    text=ammissibile,
    sort=ammissibile,
    description={Soluzione che rispetta tutti i vincoli del problema}
}

\newglossaryentry{astrazioneg}
{
    name=\glslink{astrazione}{Astrazione},
    text=astrazione,
    sort=astrazione,
    description={Applicazione del metodo logico di astrazione nella strutturazione della
    descrizione dei sistemi informatici complessi, per facilitarne la progettazione e
    manutenzione o la stessa comprensione. La pratica consiste nel presentare il
    sistema, ad esempio un pezzo di codice sorgente o uno scambio di trasmissioni di
    dati, in maniera ridotta ai soli dettagli considerati essenziali all’interesse specifico,
    ad esempio raggruppando il codice in una funzione o formalizzando un protocollo
    di comunicazione \cite{site:wiki}}
}

\newglossaryentry{businessintelligenceg}
{
    name=\glslink{businessintelligence}{Business Intelligence},
    text=Business Intelligence,
    sort=businessintelligence,
    description={Con la locuzione \textit{business intelligence}
    (\textit{BI}), ci si può solitamente riferire a: un insieme di processi aziendali per raccogliere dati ed analizzare
    informazioni strategiche, la tecnologia utilizzata per realizzare questi processi
    oppure le informazioni ottenute come risultato di questi processi \cite{site:wiki}}
}

\newglossaryentry{debugg}
{
    name=\glslink{debug}{Debug},
    text=debug,
    sort=debug,
    description={Termine che indica l’attività che
    consiste nell’individuazione e correzione da parte del programmatore di uno o più
    errori (\textit{bug}) rilevati nel \textit{software}, direttamente in fase di programmazione oppure
    a seguito della fase di \textit{testing} o dell’utilizzo finale del programma stesso \cite{site:wiki}}
}

\newglossaryentry{debuggerg}
{
    name=\glslink{debugger}{Debugger},
    text=debugger,
    sort=debugger,
    description={Programma/\textit{software} specificatamente
    progettato per l’analisi e l’eliminazione dei \textit{bug} (\textit{debugging}), ovvero errori di
    programmazione interni al codice di altri programmi \cite{site:wiki}}
}

\newglossaryentry{ereditarietag}
{
    name=\glslink{ereditarietà}{Ereditarietà},
    text=ereditarietà,
    sort=ereditarietà,
    description={Uno dei principi fondamentali della programmazione ad oggetti. In
    generale, essa rappresenta un meccanismo che consente di creare nuovi oggetti
    che siano basati su altri già definiti. Si definisce oggetto figlio 
    quello che eredita tutte o parte delle proprietà e dei metodi definiti nell’oggetto
    padre \cite{site:wiki}}
}

\newglossaryentry{ideg}
{
    name=\glslink{ide}{IDE},
    text=IDE,
    sort=ide,
    description={(\textit{Integrated Development Environment})
    \textit{Software} che aiuta gli sviluppatori nella fase di programmazione con vari strumenti
    di analisi del codice e integrazione di altre componenti \cite{site:wiki}}
}

\newglossaryentry{intellisenseg}
{
    name=\glslink{intellisense}{IntelliSense},
    text=intellisense,
    sort=intellisense,
    description={Forma di completamento automatico resa popolare da Visual Studio.
    Serve inoltre come documentazione per i
    nomi delle variabili, delle funzioni e dei metodi usando metadati. L’uso
    dell’IntelliSense è un metodo conveniente per visualizzare la descrizione delle
    funzioni, in particolar modo la lista dei loro parametri. Questa tecnologia riesce
    a velocizzare lo sviluppo del \textit{software} riducendo la quantità di \textit{input} attraverso
    la tastiera \cite{site:wiki}}
}

\newglossaryentry{logg}
{
    name=\glslink{log}{Log},
    text=log,
    sort=log,
    description={\textit{File} su cui sono registrati eventi caratteristici dell’applicazione e che fungono
    in certi casi da vero e proprio protocollo di entrata e di uscita. Ad esempio, in
    programmazione, il \textit{file} di \textit{log} evidenzia il tipo di errore e il punto nel codice
    da parte dell’\textit{IDE} \cite{site:wiki}}
}

\newglossaryentry{oltpg}
{
    name=\glslink{oltp}{OLTP},
    text=OLTP,
    sort=oltp,
    description={(\textit{Online transaction processing}) è un insieme di tecniche \textit{software} utilizzate per
    la gestione di applicazioni orientate alle transazioni \cite{site:wiki}}
}

\newglossaryentry{polimorfismog}
{
    name=\glslink{polimorfismo}{Polimorfismo},
    text=polimorfismo,
    sort=polimorfismo,
    description={Termine usato in senso generico
    per riferirsi a espressioni che possono rappresentare valori di diversi tipi (dette
    espressioni polimorfiche). In un linguaggio non tipizzato, tutte le espressioni
    sono intrinsecamente polimorfiche. Nel contesto della programmazione orientata
    agli oggetti, si riferisce al fatto che una espressione il cui tipo sia descritto da
    una classe A può assumere valori di un qualunque tipo descritto da una classe B
    sottoclasse di A (polimorfismo per inclusione) \cite{site:wiki}}
}

\newglossaryentry{refactoringg}
{
    name=\glslink{refactoring}{Refactoring},
    text=refactoring,
    sort=refactoring,
    description={Tecnica per modificare la struttura interna di porzioni di codice
    senza modificarne il comportamento esterno, applicata per migliorare alcune
    caratteristiche non funzionali del \textit{software} \cite{Addison-Wesley:software-engineering}}
}

\newglossaryentry{ricercalocaleg}
{
    name=\glslink{ricercalocale}{Ricerca locale},
    text=ricerca locale,
    sort=ricercalocale,
    description={Algoritmo che si basa sull’idea di migliorare
    una soluzione iniziale esplorandone un intorno
    opportunamente definito. Se l’ottimizzazione
    dell’intorno produce una soluzione migliorante
    il procedimento viene ripetuto
    considerando come soluzione corrente la soluzione
    appena determinata. L’algoritmo termina se non è più
    possibile trovare soluzioni miglioranti, quindi al
    raggiungimento di un ottimo locale, oppure al
    raggiungimento di un numero
    determinato di iterazioni o di un tempo massimo di esecuzione \cite{site:dispense-de-giovanni}}
}

\newglossaryentry{javag}
{
    name=\glslink{java}{Java},
    text=Java,
    sort=java,
    description={linguaggio di programmazione
    \textit{general-purpose} ad livello, basato su classi, orientato alla programmazione ad oggetti e
    progettato
    per avere il minor numero di dipendenze di implementazione
    possibile \cite{site:wiki}}
}

\newglossaryentry{linqg}
{
    name=\glslink{linq}{LINQ},
    text=LINQ,
    sort=linq,
    description={(\textit{Language-Integrated Query})
    Insieme di tecnologie basate sull'integrazione delle funzionalità di query direttamente nel linguaggio C\# \cite{site:wiki}}
}

\newglossaryentry{visualbasicg}
{
    name=\glslink{visualbasic}{Visual Basic},
    text=Visual Basic,
    sort=visualbasic,
    description={Linguaggio di programmazione \textit{event driven} creato da \textit{Microsoft} \cite{site:wiki}}
}

\newglossaryentry{C++g}
{
    name=\glslink{c++}{C++},
    text=C++,
    sort=c++,
    description={Linguaggio di programmazione
    general purpose sviluppato come evoluzione del linguaggio C inserendo la programmazione orientata agli oggetti}
}

\newglossaryentry{JUnitg}
{
    name=\glslink{junit}{JUnit},
    text=JUnit,
    sort=junit,
    description={\textit{Framework} di \textit{unit testing} per il linguaggio di programmazione \textit{Java}}
}

\newglossaryentry{aaag}
{
    name=\glslink{aaa}{Arrange-Act-Assert},
    text=AAA,
    sort=aaa,
    description={\textit{Pattern} per scrivere \textit{test} di unità che abbiano una struttura uniforme. Questo li rende anche facilmente leggibili e comprensibili}
}

\newglossaryentry{nosqlg}
{
    name=\glslink{nosql}{NoSQL},
    text=NoSQL,
    sort=nosql,
    description={Tipo di \textit{database} realizzati per modelli di dati specifici con schemi flessibili. Sono molto utilizzati per la facilità di sviluppo, le funzionalità che offrono e la scalabilità delle prestazioni \cite{site:wiki}}
}

\newglossaryentry{databaseg}
{
    name=\glslink{database}{Database},
    text=database,
    sort=database,
    description={Archivio di dati strutturato in modo da razionalizzare la gestione e l'aggiornamento delle informazioni e da permettere lo svolgimento di ricerche complesse \cite{site:def-db}}
}

\newglossaryentry{preprocessingg}
{
    name=\glslink{preprocessing}{Pre-processing},
    text=pre-processing,
    sort=preprocessing,
    description={Consiste nella manipolazione o nell'eliminazione
    dei dati prima che vengano utilizzati al fine di
    garantire o migliorare le prestazioni \cite{site:wiki}}
}

\newglossaryentry{sqlg}
{
    name=\glslink{sql}{SQL},
    text=SQL,
    sort=sql,
    description={(Structured Query Language) Linguaggio standardizzato per \textit{database} basati sul modello relazionale}
}

\newglossaryentry{milestoneg}
{
    name=\glslink{milestone}{Milestone},
    text=milestone,
    sort=milestone,
    description={Consiste in importanti traguardi intermedi nello svolgimento di un progetto}
}

\newglossaryentry{entityframeworkg}
{
    name=\glslink{entityframework}{Entity Framework},
    text=Entity Framework,
    sort=entityframework,
    description={Mappatore di \textit{database} di oggetti per l'ambiente \textit{.NET} \cite{site:docs-ef}}
}

\newglossaryentry{rdbmsg}
{
    name=\glslink{rdbms}{Relational Database Management System},
    text=RDBMS,
    sort=rdbms,
    description={Sistema \textit{software} progettato per consentire la creazione, la manipolazione e l'interrogazione efficiente di \textit{database}, per questo detto anche "gestore o motore del \textit{database}". È basato sul modello relazionale \cite{site:wiki}}
} % database di termini


%**************************************************************
% Impostazioni di graphicx
%**************************************************************
\graphicspath{{immagini/}} % cartella dove sono riposte le immagini


%**************************************************************
% Impostazioni di hyperref
%**************************************************************
\hypersetup{
    %hyperfootnotes=false,
    %pdfpagelabels,
    %draft,	% = elimina tutti i link (utile per stampe in bianco e nero)
    colorlinks=true,
    linktocpage=true,
    pdfstartpage=1,
    pdfstartview=,
    % decommenta la riga seguente per avere link in nero (per esempio per la stampa in bianco e nero)
    %colorlinks=false, linktocpage=false, pdfborder={0 0 0}, pdfstartpage=1, pdfstartview=FitV,
    breaklinks=true,
    pdfpagemode=UseNone,
    pageanchor=true,
    pdfpagemode=UseOutlines,
    plainpages=false,
    bookmarksnumbered,
    bookmarksopen=true,
    bookmarksopenlevel=1,
    hypertexnames=true,
    pdfhighlight=/O,
    %nesting=true,
    %frenchlinks,
    urlcolor=webbrown,
    linkcolor=RoyalBlue,
    citecolor=webgreen,
    %pagecolor=RoyalBlue,
    %urlcolor=Black, linkcolor=Black, citecolor=Black, %pagecolor=Black,
    pdftitle={\myTitle},
    pdfauthor={\textcopyright\ \myName, \myUni, \myFaculty},
    pdfsubject={},
    pdfkeywords={},
    pdfcreator={pdfLaTeX},
    pdfproducer={LaTeX}
}

%**************************************************************
% Impostazioni di itemize
%**************************************************************
\renewcommand{\labelitemi}{$\ast$}

%\renewcommand{\labelitemi}{$\bullet$}
%\renewcommand{\labelitemii}{$\cdot$}
%\renewcommand{\labelitemiii}{$\diamond$}
%\renewcommand{\labelitemiv}{$\ast$}


%**************************************************************
% Impostazioni di listings
%**************************************************************
\lstset{
    language=[LaTeX]Tex,%C++,
    keywordstyle=\color{RoyalBlue}, %\bfseries,
    basicstyle=\small\ttfamily,
    %identifierstyle=\color{NavyBlue},
    commentstyle=\color{Green}\ttfamily,
    stringstyle=\rmfamily,
    numbers=none, %left,%
    numberstyle=\scriptsize, %\tiny
    stepnumber=5,
    numbersep=8pt,
    showstringspaces=false,
    breaklines=true,
    frameround=ftff,
    frame=single
}


%**************************************************************
% Impostazioni di xcolor
%**************************************************************
\definecolor{webgreen}{rgb}{0,.5,0}
\definecolor{webbrown}{rgb}{.6,0,0}


%**************************************************************
% Altro
%**************************************************************

\newcommand{\omissis}{[\dots\negthinspace]} % produce [...]

% eccezioni all'algoritmo di sillabazione
\hyphenation
{
    ma-cro-istru-zio-ne
    gi-ral-din
}

\newcommand{\sectionname}{sezione}
\addto\captionsitalian{\renewcommand{\figurename}{Figura}
    \renewcommand{\tablename}{Tabella}}

\newcommand{\glsfirstoccur}{\ap{{[g]}}}

\newcommand{\intro}[1]{\emph{\textsf{#1}}}

%**************************************************************
% Environment per ``rischi''
%**************************************************************
\newcounter{riskcounter}                % define a counter
\setcounter{riskcounter}{0}             % set the counter to some initial value

%%%% Parameters
% #1: Title
\newenvironment{risk}[1]{
    \refstepcounter{riskcounter}        % increment counter
    \par \noindent                      % start new paragraph
    \textbf{\arabic{riskcounter}. #1}   % display the title before the 
    % content of the environment is displayed 
}{
    \par\medskip
}

\newcommand{\riskname}{Rischio}

\newcommand{\riskdescription}[1]{\textbf{\\Descrizione:} #1.}

\newcommand{\risksolution}[1]{\textbf{\\Soluzione:} #1.}

%**************************************************************
% Environment per ``use case''
%**************************************************************
\newcounter{usecasecounter}             % define a counter
\setcounter{usecasecounter}{0}          % set the counter to some initial value

%%%% Parameters
% #1: ID
% #2: Nome
\newenvironment{usecase}[2]{
    \renewcommand{\theusecasecounter}{\usecasename #1}  % this is where the display of 
    % the counter is overwritten/modified
    \refstepcounter{usecasecounter}             % increment counter
    \vspace{10pt}
    \par \noindent                              % start new paragraph
    {\large \textbf{\usecasename #1: #2}}       % display the title before the 
    % content of the environment is displayed 
    \medskip
}{
    \medskip
}

\newcommand{\usecasename}{UC}

\newcommand{\usecaseactors}[1]{\textbf{\\Attori Principali:} #1. \vspace{4pt}}
\newcommand{\usecasepre}[1]{\textbf{\\Precondizioni:} #1. \vspace{4pt}}
\newcommand{\usecasedesc}[1]{\textbf{\\Descrizione:} #1. \vspace{4pt}}
\newcommand{\usecasepost}[1]{\textbf{\\Postcondizioni:} #1. \vspace{4pt}}
\newcommand{\usecasealt}[1]{\textbf{\\Scenario Alternativo:} #1. \vspace{4pt}}

%**************************************************************
% Environment per ``namespace description''
%**************************************************************

\newenvironment{namespacedesc}{
    \vspace{10pt}
    \par \noindent                              % start new paragraph
    \begin{description}
        }{
    \end{description}
    \medskip
}

\newcommand{\classdesc}[2]{\item[\textbf{#1:}] #2}
