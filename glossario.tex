
%**************************************************************
% Acronimi
%**************************************************************
%\newacronym[description={\glslink{apig}{Application Program Interface}}]
%    {api}{API}{Application Program Interface}
%
%\newacronym[description={\glslink{umlg}{Unified Modeling Language}}]
%    {uml}{UML}{Unified Modeling Language}

%**************************************************************
% Glossario
%**************************************************************
\newglossaryentry{apig}
{
    name=\glslink{api}{API},
    text=Application Program Interface,
    sort=api,
    description={(\emph{Application Programming Interface API})
    intermediario software che permette a due
    applicazioni non correlate di comunicare tra loro}
}

\newglossaryentry{umlg}
{
    name=\glslink{uml}{UML},
    text=UML,
    sort=uml,
    description={(\emph{Unified Model Language})
    standard utilizzato nell'ingegneria del software
    per descrivere un sistema informatico, consentendo ai
    vari ruoli (sviluppatore, tester, analista,ecc.) di
    comunicare tramite lo stesso linguaggio. Si avvale di
    diversi tipi di diagrammi, statici e dinamici,
    per fornire diverse viste di uno stesso sistema}
}

\newglossaryentry{erpg}
{
    name=\glslink{erp}{ERP},
    text=ERP,
    sort=erp,
    description={(\emph{Enterprise Resource Planning})
    tipologia di software che integra tutti i processi
    di business rilevanti di un'azienda e tutte le funzioni
    aziendali, ad esempio vendite, acquisti, gestione magazzino,
    finanza, contabilità..}
}

\newglossaryentry{.netg}
{
    name=\glslink{.netframework}{.NET Framework},
    text=.NET Framework,
    sort=.netframework,
    description={Ambiente di esecuzione runtime
    della piattaforma tecnologica \textit{.NET} in cui
    vengono gestite le applicazioni destinate allo stesso
    \textit{.NET Framework}.
    Disponibile solo su \textit{Windows}}
}

\newglossaryentry{devexpressg}
{
    name=\glslink{devexpress}{DevExpress},
    text=DevExpress,
    sort=devexpress,
    description={\textit{Framework} utile per lo sviluppo
    di applicazioni desktop}
}

\newglossaryentry{serverconsolidationg}
{
    name=\glslink{serverconsolidation}{Server Consolidation},
    text=server consolidation,
    sort=serverconsolidation,
    description={È un approccio all'utilizzo efficiente
    delle risorse dei server dei computer al fine di
    ridurre il numero totale di server o posizioni di
    server richiesti da un'organizzazione}
}

\newglossaryentry{ergdisg}
{
    name=\glslink{ergdis}{ERGDIS},
    text=ERGDIS,
    sort=ergdis,
    description={Software proprietario dell'azienda \textit{\myCompany}}
}

\newglossaryentry{stakeholdersg}
{
    name=\glslink{stakeholder}{Stakeholders},
    text=stakeholders,
    sort=stakeholders,
    description={Coloro che a vario titolo hanno influenza su un
    prodotto o su un progetto}
}

\newglossaryentry{windowsformg}
{
    name=\glslink{windowsform}{Windows form},
    text=Windows form,
    sort=windowsform,
    description={Applicazione basata su eventi
    supportata da \textit{.NET Framework} di \textit{Microsoft}}
}

\newglossaryentry{ricercaoperativag}
{
    name=\glslink{ricercaoperativa}{Ricerca operativa},
    text=ricerca operativa,
    sort=ricercaoperativa,
    description={Scienza che fornisce strumenti matematici di
    supporto alle attività decisionali in cui occorre gestire
    ecoordinare attività e risorse limitate}
}

\newglossaryentry{ammissibileg}
{
    name=\glslink{ammissibile}{Ammissibile},
    text=ammissibile,
    sort=ammissibile,
    description={Soluzione che rispetta tutti i vincoli del problema}
}

\newglossaryentry{astrazioneg}
{
    name=\glslink{astrazione}{Astrazione},
    text=astrazione,
    sort=astrazione,
    description={Applicazione del metodo logico di astrazione nella strutturazione della
    descrizione dei sistemi informatici complessi, per facilitarne la progettazione e
    manutenzione o la stessa comprensione. La pratica consiste nel presentare il
    sistema, ad esempio un pezzo di codice sorgente o uno scambio di trasmissioni di
    dati, in maniera ridotta ai soli dettagli considerati essenziali all’interesse specifico,
    ad esempio raggruppando il codice in una funzione o formalizzando un protocollo
    di comunicazione}
}

\newglossaryentry{businessintelligenceg}
{
    name=\glslink{businessintelligence}{Business Intelligence},
    text=Business Intelligence,
    sort=businessintelligence,
    description={Con la locuzione \textit{business intelligence}
    (\textit{BI}) ci si può solitamente riferire a: un insieme di processi aziendali per raccogliere dati ed analizzare
    informazioni strategiche, la tecnologia utilizzata per realizzare questi processi
    oppure le informazioni ottenute come risultato di questi processi}
}

\newglossaryentry{debugg}
{
    name=\glslink{debug}{Debug},
    text=debug,
    sort=debug,
    description={Termine che indica l’attività che
    consiste nell’individuazione e correzione da parte del programmatore di uno o più
    errori (\textit{bug}) rilevati nel \textit{software}, direttamente in fase di programmazione oppure
    a seguito della fase di \textit{testing} o dell’utilizzo finale del programma stesso}
}

\newglossaryentry{debuggerg}
{
    name=\glslink{debugger}{Debugger},
    text=debugger,
    sort=debugger,
    description={Programma/\textit{software} specificatamente
    progettato per l’analisi e l’eliminazione dei \textit{bug} (\textit{debugging}), ovvero errori di
    programmazione interni al codice di altri programmi}
}

\newglossaryentry{ereditarietag}
{
    name=\glslink{ereditarietà}{Ereditarietà},
    text=ereditarietà,
    sort=ereditarietà,
    description={Uno dei Principio fondamentale della programmazione ad oggetti. In
    generale, essa rappresenta un meccanismo che consente di creare nuovi oggetti
    che siano basati su altri già definiti. Si definisce oggetto figlio 
    quello che eredita tutte o parte delle proprietà e dei metodi definiti nell’oggetto
    padre}
}

\newglossaryentry{ideg}
{
    name=\glslink{ide}{IDE},
    text=ide,
    sort=ide,
    description={Ambiente di sviluppo integrato
    (\textit{Integrated Development Environment}) è un
    \textit{software} che aiuta gli sviluppatori nella fase di programmazione con vari strumenti
    di analisi del codice e integrazione di altre componenti}
}

\newglossaryentry{intellisenseg}
{
    name=\glslink{intellisense}{IntelliSense},
    text=intellisense,
    sort=intellisense,
    description={Forma di completamento automatico resa popolare da Visual Studio
    Serve inoltre come documentazione per i
    nomi delle variabili, delle funzioni e dei metodi usando metadati. L’uso
    dell’IntelliSense è un metodo conveniente per visualizzare la descrizione delle
    funzioni, in particolar modo la lista dei loro parametri. Questa tecnologia riesce
    a velocizzare lo sviluppo del \textit{software} riducendo la quantità di \textit{input} attraverso
    la tastiera}
}

